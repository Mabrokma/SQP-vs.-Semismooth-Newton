\chapter{Einf�hrung}

Die nichtlineare Optimierung ist ein bedeutendes Gebiet der Mathematik.
Sie findet immer wieder Anwendungen in der schwierigen Problemen der Technik und
der Wirtschaft. Es wurden viele Verfahren entwickelt, um nichtlineare
Optimierungsprobleme zu l�sen. In dieser Arbeit werden zwei Verfahren, das
halbglatte Newton-Verfahren und das SQP-Verfahren, betrachtet und verglichen.

Das SQP-Verfahren geh�rt zu den bekanntesten Verfahren der nichtlinearen
Optimierung. Es wurde schon seit den 60er Jahren entwickelt und wurde in vielen
Optimierungsproblemen als Standardwerkzeug angewendet sowie weiterentwickelt.
Das halbglatte Newton-Verfahren ist weniger bekannt als das SQP-Verfahren. Es basiert aber auf
das bekannte Newton-Verfahren.

% TODO tell more about SQP and Semismooth-Newton

\section{Optimierungsprobleme}

Allgemein ist die Aufgabenstellung der nichtlinearen Optimierung wie folgt
definiert:
\begin{equation}
  \min_{x \in \F} f(x) \label{P}
\end{equation}

Die Funktion $f:D \subseteq \R^n \rightarrow \R$ ist die sogennante Zielfunktion.
$D$ sei der Definitionsbereich von $f$.
$\F$ sei eine Teilmenge von $D$, die man als L�sungsmenge bezeichnet.
Alle Elemente von $\F$ werden als zul�ssige Punkte bezeichnet.
$\F$ wird durch Nebenbedingungen definiert.

Man kann hierbei den Unterschied zwischen der linearen Optimierung und
der nichtlinearen Optimierung erkennen.
Bei der linearen Optimierung muss die Zielfunktion linear sein
(d.h. die Zielfunktion besitzt die Form $f(x) = c^T x$, $c \in \R^n$)
und die Nebenbedingungen sind durch lineare Gleichungssysteme oder
Ungleichungssyteme definiert.
Bei der nichtlinearen Optimierung gibt es dagegen keine Einschr�nkung,
wie die Zielfunktion und die Nebenbedingungen aussehen sollen.
F�r die lineare Optimierung ist ein in der Praxis sehr effizientes Verfahren
bekannt. Aber die Verfahren der nichtlinearen Optimierung sind auch f�r die
linearen Optimierungsprobleme anwendbar.
Umgekehrt ist es jedoch nicht m�glich.

Ein einfaches Beispiel nichtlinearer Optimierung ist das Problem
\[
  \min_{x \in \R} (x-1)^2.
\]

Falls $\F = D$ gilt,
dann bezeichnet man das Optimierungsproblem als unrestringiert.
Es besitzt also keine Nebenbedingungen.
Ansonsten hei�t es ein restingiertes Optimierungsproblem.

Ein Punkt $\xopt \in \F$ hei�t globale L�sung des Problems (\ref{P}),
wenn
\begin{equation}
  f(\xopt) \leq f(x) \qquad \forall x \in \F
\end{equation}
gilt.
Ein Punkt $\xopt \in \F$ hei�t strikte globale L�sung des Problems
(\ref{P}), wenn
\begin{equation}
  f(\xopt) < f(x) \qquad \forall x \in \F\backslash\{\xopt\}
\end{equation}
gilt.
Ein Punkt $\xopt \in \F$ hei�t lokale L�sung des Problems
(\ref{P}), wenn f�r eine Umgebung $U(\xopt)$ von $\xopt$
\begin{equation}
  f(\xopt) \leq f(x) \qquad \forall x \in U(\xopt) \cap \F
\end{equation}
gilt.
Ein Punkt $\xopt \in \F$ hei�t strikte lokale L�sung des Problems
(\ref{P}), wenn f�r eine Umgebung $U(\xopt)$ von $\xopt$
\begin{equation}
  f(\xopt) < f(x) \qquad \forall x \in U(\xopt) \cap \F\backslash\{\xopt\}
\end{equation}
gilt.
Eine Umgebung von $\xopt$ ist einfach eine offene Menge, die $\xopt$ beinhaltet.

% x^4 \cos(1/x) + 2x^4 ?

Wegen dieser Definitionen kommt die Unterscheidung zwischen der globalen
und der lokalen Optimierung. Globale Optimierung versucht, globale L�sungen
zu finden. Lokale Optimierung versucht dagegen, lokale L�sungen zu finden.
Viele Verfahren finden lokale L�sungen, die jedoch nicht unbedingt
globale L�sungen sind.

% TODO?: Konvexit�t

% TODO: Iterative Methode: \[ f(x^{k+1}) < f(x^k) \]

% TODO: Unrestringerte Optimierungsprobleme: Theorie und Verfahren

Wir betrachten nun das unrestringierte Optimierungsproblem
\begin{equation}
  \min_{x \in D} f(x). \label{PU}
\end{equation}

Notwendige Bedingung erster Ordnung:

Seien $\xopt$ eine lokale L�sung des Problems (\ref{PU}) und $f$ einmal
stetig differenzierbar in einer Umgebung von $\xopt$, dann gilt
\begin{equation}
  \nabla f(\xopt) = 0
\end{equation}

Notwendige Bedingung zweiter Ordnung:

Sei $\xopt$ eine lokale L�sung des Problems (\ref{PU}) und es existiere
die Hessian-Matrix $H_f$, die stetig in einer Umgebung von $\xopt$ sei,
dann gilt
\begin{equation}
  \nabla f(\xopt) = 0
  \text{ und }
  H_f(\xopt) \text{ ist positiv semidefinit.}
\end{equation}

Hinreichende Bedingung zweiter Ordnung:

Es existiere die Hessian-Matrix $H_f$, die stetig in einer Umgebung von $\xopt$
sei, und es gelte $\nabla f(\xopt) = 0$ und $H_f(\xopt)$ sei positiv definit,
dann ist $\xopt$ eine strikte L�sung des Problems (\ref{PU}).


% TODO: Approximation der Gradient und Hesse-Matrix?
