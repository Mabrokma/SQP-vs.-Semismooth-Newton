\chapter{Einf�hrung}
%\section{}

Die Mathematik hat schon immer eine enge Beziehung mit Optimierung.
Viele Theorien wurden entwickelt, um eine Optimierungsaufgabe zu l�sen.
In der Industrie werden immer wieder gefragt, wie man einen maximalen Gewinn
erzielen kann.
Oft ist auch die Frage wie man etwas minimieren kann,
wie zum Beispiel die gebrauchte Stoffe.
Optimierung spielt immer wieder eine wichtige Rolle,
wenn es um eine Entscheidung geht.

Die erste Optimierungsaufgabe? (Euler? Lagrange?)

Zur Optimierung geh�ren folgende:
\begin{itemize}
\item Zielfunktion
\item Variablen
\item Bedingungen
\end{itemize}

Modellierung (kann zu kompliziert sein und sogar nicht l�sbar => vereinfachen)

Optimierungsalgorithmen



Optimierung

\[min_{x \in R^n} f(x)\]

$x$ ist die Variable
$f$ ist die Zielfunktion, in dem Fall ist sie eine Skalarfunktion, also hat $R$ als Wertebereich

Beispiel (mit Bild)
Ein einfaches Beispiel: $f(x) = (x-1)^2$



Minima?

Globale Optimierung vs. Lokale Optimierung
Restringiert oder Unrestringiert
Konvexit�t?
Unrestringierte Optimierung?


Approximation der Gradient und Hesse-Matrix?

