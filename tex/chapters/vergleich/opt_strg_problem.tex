\begin{testproblem}
(Steuerungsproblem, vgl. Beispiel~6.2.6 in \cite[S.~216~f.]{alt})\\
Es ist ein Produktionsplanungsproblem mit beschr�nktem Lager zu betrachten.
In einem Zeitraum $[0,T]$ mit $T > 0$
soll eine Produktion so gesteuert werden,
dass die Gesamtkosten minimal sind.
Seien
\begin{itemize}
  \item $z(t)$ die Anzahl der zur Zeit $t$ am Lager vorhandenen Produkte,
  \item $r(t)$ die Nachfrage nach den Produkten zur Zeit $t$ und
  \item $u(t)$ die Produktionsrate zur Zeit $t$.
\end{itemize}
Die �nderung der verf�gbaren Lagermenge $z(t)$
sei durch die Lagerbilanzgleichung
\begin{equation}
  \dot{z}(t) = u(t) - r(t) \qquad \forall t \in [0,T]
\end{equation}
mit dem Anfangslagerbestand $z(0) = a \geq 0$ gegeben.
Die Produktions- und Lagehaltungskosten zur Zeit $t$ betragen
$\frac{1}{2} \rho_u u(t)^2 + \rho_z z(t)$
mit Konstanten $\rho_u, \rho_z > 0$.
Durch den Verkauf des Lagerrestbestandes $z(T)$
kommt ein Gewinn $\sigma z(T)$ mit $\sigma \geq 0$.
Dabei sind $u(t)$ und $z(t)$ zu bestimmen,
wobei die Lagerkapazit�t beschr�nkt ist,
d.\,h. $z(t) \leq \zeta$ f�r alle $t \in [0,T]$ mit $\zeta > 0$.
Damit ergibt sich das folgende zeitkontinuierliche Problem: 
\begin{gather}
  \min\: f(z,u) :=
    \int_0^T \left( \tfrac{1}{2} \rho_u u(t)^2 + \rho_z z(t) \right) d\!t
    - \sigma z(T) \label{prob:opt_strg_prob_produktionsplannung} \\
  \nb \quad \dot{z}(t) = u(t) - r(t) \quad \forall t \in [0,T] \notag \\
    z(0) = a \notag \\
    0 \leq z(t) \leq \zeta \quad \forall t \in [0,T] \notag \\
    0 \leq u(t) \quad \forall t \in [0,T]. \notag
\end{gather}
Eine Diskretisierung soll durchgef�hrt werden,
damit das Problem numerisch gel�st werden kann.
Mit $N \in \N$ und $\tau := \frac{T}{N}$
wird das Intervall $[0,T]$ in $N$ Teilintervalle
$[t_0,t_1], \ldots, [t_{N-1},t_N]$ unterteilt,
wobei $t_i = i\tau$ f�r alle $i = 0,\ldots,N$.
Die Produktionsrate $u(t)$ wird durch
eine st�ckweise konstante Funktion $\hat{u}(t)$ mit
\begin{equation}
  \hat{u}(t) := u_i \qquad
    \forall\, t \in [t_i, t_{i+1}[,\:
    i=0,\ldots,N-1
\end{equation}
approximiert
und die Lagermenge $z(t)$ durch
eine lineare Interpolationsfunktion $\hat{z}(t)$
mit den St�tzstellen
$(t_0,a),(t_1,z_1),\ldots,(t_N,z_N)$, d.\,h.
\begin{equation}
  \hat{z}(t) := z_i + \frac{1}{\tau}(z_{i+1} - z_i)(t - t_i) \qquad
    \forall\, t \in [t_i, t_{i+1}[, \:
    i=0,\ldots,N-1
\end{equation}
mit $z_0 := a$.
Die diskrete Version des
Problems~\eqref{prob:opt_strg_prob_produktionsplannung}
l�sst sich dann wie folgt formulieren.
\begin{gather}
  \min\: \frac{1}{2} \tau \rho_u \sum_{i=0}^{N-1} u_i^2
    + \tau \rho_z \sum_{i=1}^{N} z_i
    - \sigma z_N \label{prob:opt_strg_prob_produktionsplan_diskr}\\
  \nb \quad z_{i+1} - z_i - \tau (u_i - r_i) = 0
    \quad \forall i = 0,\ldots,N-1 \notag \\
    0 \leq z_i \leq \zeta \quad \forall i = 1,\ldots,N \notag \\
    0 \leq u_i \quad \forall i = 0,\ldots,N-1, \notag
\end{gather}
wobei $r_i := r(t_i)$ f�r $i=0,\ldots,N-1$.
Das Problem~\eqref{prob:opt_strg_prob_produktionsplan_diskr}
ist ein Problem vom Typ~\eqref{prob:quad_opt_prob_mit_lin_ungl_nebenbed}
und $x := (u_0,z_1,u_1,z_2,\ldots,u_{N-1},z_{N})^T$ sei
als die Reihenfolge der Variablen gew�hlt.
Konkrete Daten f�r das Testproblem seien
\begin{gather}
  N = 100,\ T = 12,\ \rho_u = 0.2,\ \rho_z = 0.1,\\
  \sigma = 1.5,\ a = 5,\ \zeta = 6,\ r(t) = 10 + 5 \sin{t}.
\end{gather}
Der Startvektor $x^0 \in \R^{2N}$ sei durch
\begin{equation}
  x^0_{2i+1} := r_i, \quad x^0_{2i+2} := a
  \qquad \forall i=0,\ldots,N-1
\end{equation}
definiert.
\end{testproblem}