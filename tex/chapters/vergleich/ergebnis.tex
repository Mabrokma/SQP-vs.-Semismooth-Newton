Das halbglatte Newton-Verfahren und das SQP-Verfahren
wurden in der Programmiersprache MATLAB implementiert.
Bei allen im Unterkapitel~\ref{sec:testprobleme}
vorgestellten Testproblemen waren
die implementierten Programme in der Lage,
die angegebene L�sung zu finden.
Die entsprechenden Programme bezeichnen wir
hier jeweils als SSN und SQP.

Die folgenden zwei Tabellen zeigen die Ergebnisse des Tests.
Die beiden Programme wurden 100 Mal
auf jedes Testproblem angewendet
und der Durschnitt der Laufzeiten ist
als die ermittelte Laufzeit zu verstehen.
Die Bezeichnung $T$ steht f�r die Laufzeit
des Programms zur L�sung des Problems in Millisekunden
und \#\emph{It} bezeichnet die Anzahl der ben�tigten Iterationen.
Als Abbruchschranke f�r die beiden Programme
wurde $\varepsilon = 10^{-5}$ gew�hlt.
Die Abbildung~\ref{fig:vergleich_der_laufzeiten}
stellt die Laufzeiten der beiden Programme graphisch dar.

\begin{table}[h]
\centering
\begin{tabular*}{0.75\linewidth}{@{\extracolsep{\fill}}crrrr}
  \toprule
  \emph{Nr.} & $T_\text{\emph{SSN}}$\,\emph{(ms)} &
    $T_\text{\emph{SQP}}$\,\emph{(ms)} &
    \#\emph{It}$_\text{\emph{SSN}}$ & \#\emph{It}$_\text{\emph{SQP}}$ \\
  \midrule
  1 & 1.38 & 1.94 & 2 & 1 \\ % problem_v_norm
  2 & 3.87 & 6.94 & 6 & 5 \\ % problem_v_rosenbrock
  3 & 4.80 & 8.88 & 7 & 6 \\ % problem_v_himmelblau
  4 & 9.12 & 36.33 & 15 & 29 \\ % problem_v_bazaraa_shetty
  5 & 19.82 & 24.85 & 9 & 8 \\ % problem_v_beale
  6 & 35.54 & 64.52 & 48 & 47 \\ % problem_v_exp
  7 & 171.86 & 195.18 & 27 & 26 \\ % problem_v_colville
  8 & 148.26 & 210.62 & 36 & 31 \\ % problem_v_dixon
  9 & 2.56 & 6.17 & 3 & 1 \\ % problem_v_norm_1
  10 & 5.78 & 13.16 & 9 & 6 \\ % problem_v_rosenbrock_1
  11 & 6.22 & 21.10 & 9 & 10 \\ % problem_v_himmelblau_1
  12 & 3.98 & 5.30 & 6 & 3 \\ % problem_v_bazaraa_shetty_1
  13 & 22.27 & 21.36 & 10 & 5 \\ % problem_v_beale_1
  14 & 50.24 & 91.33 & 49 & 46 \\ % problem_v_exp_1
  15 & 70.10 & 85.54 & 11 & 10 \\ % problem_v_colville_1
  16 & 41.64 & 97.56 & 10 & 8 \\ % problem_v_dixon_1
  17 & 15.24 & 18.42 & 3 & 2 \\ % problem_v_lin_regression
  18 & 141.55 & 151.97 & 14 & 13 \\ % problem_v_quad_regression
  19 & 75.10 & 87.43 & 21 & 20 \\ % problem_v_nichtlin_regression
  \bottomrule
\end{tabular*}
\caption{Ergebnisse der Programme f�r
Testprobleme~\ref{test_prob:prob_v_quad_func}
bis~\ref{test_prob:nichtlin_regres_exp}}
\end{table}

\begin{table}[h]
\centering
\begin{tabular*}{0.75\linewidth}{@{\extracolsep{\fill}}crrrr}
  \toprule
  \emph{Nr.} & $T_\text{\emph{SSN}}$\,\emph{(ms)} &
    $T_\text{\emph{SQP}}$\,\emph{(ms)} &
    \#\emph{It}$_\text{\emph{SSN}}$ & \#\emph{It}$_\text{\emph{SQP}}$ \\
  \midrule
  20 & 1.05 & 2.27 & 2 & 1 \\ % problem_A_example_16_2_nocedal_wright
  21 & 1.08 & 2.31 & 2 & 1 \\ % problem_A_simple_example
  22 & 1.03 & 2.22 & 2 & 1 \\ % problem_A_huang_aggerwal_hs28
  23 & 1.07 & 2.34 & 2 & 1 \\ % problem_A_huang_aggerwal_miele_hs48
  24 & 1.05 & 2.30 & 2 & 1 \\ % problem_A_huang_aggerwal_hs51
  25 & 1.10 & 2.35 & 2 & 1 \\ % problem_A_miele_hs52
  26 & 2.21 & 3.81 & 2 & 1 \\ % problem_Av_betts_miele_hs53
  27 & 20.07 & 22.30 & 3 & 2 \\ % problem_A_huang_aggerwal_hs49
  28 & 65.47 & 73.87 & 9 & 8 \\ % problem_A_huang_aggerwal_hs50
  29 & 2.12 & 5.72 & 3 & 1 \\ % problem_G_example_with_diamond_area
  30 & 2.16 & 5.53 & 3 & 1 \\ % problem_Gv_example_16_4_nocedal_wright
  31 & 2.78 & 9.62 & 3 & 1 \\ % problem_G_example_13_2_antoniou_lu
  32 & 2.92 & 3.77 & 4 & 1 \\ % problem_Gv_betts_hs21
  33 & 2.01 & 4.36 & 3 & 1 \\ % problem_Gv_beale_hs35
  34 & 3.97 & 7.59 & 5 & 1 \\ % problem_Gv_murtagh_sargent_hs76
  35 & 798.02 & 509.44 & 14 & 1 \\ % problem_AG_opt_ctrl
  \bottomrule
\end{tabular*}
\caption{Ergebnisse der Programme f�r
Testprobleme~\ref{test_prob:prob_A_nocedal}
bis~\ref{test_prob:produktionsplanungsprob}}
\end{table}

Wie aus den Tabellen zu entnehmen ist,
hat SSN eine bessere Leistung bez�glich der Laufzeit
bei den Testproblemen mit weniger als 10 Variablen geliefert.
Beim Testproblem~\ref{test_prob:produktionsplanungsprob}
mit 200~Variablen war SQP schneller.

Bei den meisten Testproblemen hat SQP
wenigere Anzahl von Iterationen als SSN ben�tigt.
Theoretisch ist das auch zu erwarten,
weil nach Satz~\ref{satz:konvergenz_seq_quad_prog}
eine quadratische Konvergenz bei SQP-Verfahren im besten Fall vorkommen kann,
bei halbglattem Newton-Verfahren ist es dagegen nach
Satz~\ref{satz:konvergenz_halbglattes_newton} nur superlineare Konvergenz.
Besonders bei Testproblemen mit quadratischer Zielfunktion
hat SQP nur eine Iterationen beansprucht.

Nur bei zwei F�llen hat SQP mehr Iterationen gebraucht,
damit die Abbruchbedingung erf�llt wurde.
Wenn man aber die Iterationen genauer betrachtet,
h�tte man eigentlich das Programm schon
bei wenigeren Iterationen abbrechen k�nnen,
weil die berechnete L�sung da schon erreicht wurde.

\begin{figure}[h]
\centering
\begin{tikzpicture}[yscale=0.04,xscale=0.3]
  \draw[->] (1,0) -- (35.5,0) node [right] {\emph{Nr.}};
  \foreach \x in {2,6,...,34}
    \draw (\x,0.7) -- (\x,-0.7) node [below] {\x};
  \draw[->] (1,0) -- (1,222) node [right] {$T$\emph{(ms)}};
  \foreach \y in {0,50,...,200}
    \draw (1.2,\y) -- (0.8,\y) node [left] {\y};
  \draw[color=blue] plot file {images/sqp_results.table};
  \draw[color=red,dashed] plot file {images/ssn_results.table};
  \draw[color=blue] (25,180) -- (29,180) node [right] {SQP};
  \draw[color=red,dashed] (25,167) -- (29,167) node [right] {SSN};
\end{tikzpicture}
\caption{Laufzeiten der Programme f�r
Testprobleme~\ref{test_prob:prob_v_quad_func}
bis~\ref{test_prob:prob_Gv_hs76}}
\label{fig:vergleich_der_laufzeiten}
\end{figure}
