Bei allen vorgestellten Testproblemen sind
die beiden Verfahren in der Lage,
die L�sung zu finden.
Die folgenden Tabelle zeigen die Ergebnisse
der beiden Verfahren f�r die Testprobleme.
Die beiden Verfahren wurden 100 Mal
auf die verschiedenen Probleme angewendet
und der Durschnitt der Laufzeiten ist
als die ermittelte Laufzeit zu verstehen. 
Die Bezeichnung T$_\text{SSN}$ bzw. T$_\text{SQP}$
steht f�r die Laufzeit des halbglatten Newton-Verfahrens
bzw. des SQP-Verfahrens zur L�sung des Problems
und \#It bezeichnet die Anzahl der Iterationen.
Als Abbruchschranke f�r die beiden Verfahren
werden $\varepsilon = 0.00001$ gew�hlt.
Die Abbildung~\ref{fig:vergleich_der_laufzeiten}
stellt die Laufzeiten der beiden Verfahren graphisch dar.

\begin{table}[h]
\centering
\begin{tabular*}{0.75\linewidth}{@{\extracolsep{\fill}}crrrr}
  \toprule
  Nr. & T$_\text{SSN}$\,(ms) & T$_\text{SQP}$\,(ms)
    & \#It$_\text{SSN}$ & \#It$_\text{SQP}$ \\
  \midrule
  1 & 1.38 & 1.94 & 2 & 1 \\ % problem_v_norm
  2 & 3.87 & 6.94 & 6 & 5 \\ % problem_v_rosenbrock
  3 & 4.80 & 8.88 & 7 & 6 \\ % problem_v_himmelblau
  4 & 17.61 & 35.68 & 30 & 29 \\ % problem_v_bazaraa_shetty
  5 & 19.82 & 24.85 & 9 & 8 \\ % problem_v_beale
  6 & 35.54 & 64.52 & 48 & 47 \\ % problem_v_exp
  7 & 171.86 & 195.18 & 27 & 26 \\ % problem_v_colville
  8 & 139.26 & 218.51 & 34 & 33 \\ % problem_v_dixon
  9 & 2.56 & 6.17 & 3 & 1 \\ % problem_v_norm_1
  10 & 5.78 & 13.16 & 9 & 6 \\ % problem_v_rosenbrock_1
  11 & 6.22 & 21.10 & 9 & 10 \\ % problem_v_himmelblau_1
  12 & 3.98 & 5.30 & 6 & 3 \\ % problem_v_bazaraa_shetty_1
  13 & 22.27 & 21.36 & 10 & 5 \\ % problem_v_beale_1
  14 & 50.24 & 91.33 & 49 & 46 \\ % problem_v_exp_1
  15 & 70.10 & 85.54 & 11 & 10 \\ % problem_v_colville_1
  16 & 41.64 & 97.56 & 10 & 8 \\ % problem_v_dixon_1
  17 & 15.24 & 18.42 & 3 & 2 \\ % problem_v_lin_regression
  18 & 141.55 & 151.97 & 14 & 13 \\ % problem_v_quad_regression
  19 & 75.10 & 87.43 & 21 & 20 \\ % problem_v_nichtlin_regression
  \bottomrule
\end{tabular*}
\caption{Ergebnisse der Programme f�r Testprobleme 1 bis 19}
\end{table}

\begin{table}[h]
\centering
\begin{tabular*}{0.75\linewidth}{@{\extracolsep{\fill}}crrrr}
  \toprule
  Nr. & T$_\text{SSN}$\,(ms) & T$_\text{SQP}$\,(ms)
    & \#It$_\text{SSN}$ & \#It$_\text{SQP}$ \\
  \midrule
  20 & 1.05 & 2.27 & 2 & 1 \\ % problem_A_example_16_2_nocedal_wright
  21 & 1.08 & 2.31 & 2 & 1 \\ % problem_A_simple_example
  22 & 1.03 & 2.22 & 2 & 1 \\ % problem_A_huang_aggerwal_hs28
  23 & 1.07 & 2.34 & 2 & 1 \\ % problem_A_huang_aggerwal_miele_hs48
  24 & 1.05 & 2.30 & 2 & 1 \\ % problem_A_huang_aggerwal_hs51
  25 & 1.10 & 2.35 & 2 & 1 \\ % problem_A_miele_hs52
  26 & 2.21 & 3.81 & 2 & 1 \\ % problem_Av_betts_miele_hs53
  27 & 20.07 & 22.30 & 3 & 2 \\ % problem_A_huang_aggerwal_hs49
  28 & 65.47 & 73.87 & 9 & 8 \\ % problem_A_huang_aggerwal_hs50
  \bottomrule
\end{tabular*}
\caption{Ergebnisse der Programme f�r Testprobleme 20 bis 28}
\end{table}

\begin{table}[h]
\centering
\begin{tabular*}{0.75\linewidth}{@{\extracolsep{\fill}}crrrr}
  \toprule
  Nr. & T$_\text{SSN}$\,(ms) & T$_\text{SQP}$\,(ms)
    & \#It$_\text{SSN}$ & \#It$_\text{SQP}$ \\
  \midrule
  29 & 2.12 & 5.72 & 3 & 1 \\ % problem_G_example_with_diamond_area
  30 & 2.16 & 5.53 & 3 & 1 \\ % problem_Gv_example_16_4_nocedal_wright
  31 & 2.78 & 9.62 & 3 & 1 \\ % problem_G_example_13_2_antoniou_lu
  32 & 2.92 & 3.77 & 4 & 1 \\ % problem_Gv_betts_hs21
  33 & 2.01 & 4.36 & 3 & 1 \\ % problem_Gv_beale_hs35
  34 & 3.97 & 7.59 & 5 & 1 \\ % problem_Gv_murtagh_sargent_hs76
  \bottomrule
\end{tabular*}
\caption{Ergebnisse der Programme f�r Testprobleme 29 bis 34}
\end{table}


\begin{figure}[h]
\centering
\begin{tikzpicture}[yscale=0.04,xscale=0.3]
  \draw[->] (1,0) -- (35.5,0) node [right] {Nr.};
  \foreach \x in {2,6,...,34}
    \draw (\x,0.7) -- (\x,-0.7) node [below] {\x};
  \draw[->] (1,0) -- (1,222) node [right] {T\,(ms)};
  \foreach \y in {0,50,...,200}
    \draw (1.2,\y) -- (0.8,\y) node [left] {\y};
  \draw[color=blue] plot file {images/sqp_results.table};
  \draw[color=red,dashed] plot file {images/ssn_results.table};
  \draw[color=blue] (25,180) -- (29,180) node [right] {SQP};
  \draw[color=red,dashed] (25,167) -- (29,167) node [right] {SSN};
\end{tikzpicture}
\caption{Laufzeiten der Programme f�r Testprobleme 1 bis 34}
\label{fig:vergleich_der_laufzeiten}
\end{figure}
