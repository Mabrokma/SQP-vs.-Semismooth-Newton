
\begin{definition}
\emph{(Quadratische Optimierungsprobleme mit linearen
Gleichungsnebenbedingungen)}
\begin{align}
  \min_{x \in \R^n}\ & \frac{1}{2} \langle Qx,x \rangle + \langle q,x \rangle
    \tag{QLG} \label{prob:quad_opt_prob_mit_lin_gl_nebenbed} \\
  \nb & Ax = b \notag
\end{align}
Dabei sei $Q$ eine symmetrische $(n \times n)$-Matrix und $q \in \R^n$.
$A$ sei eine $(m \times n)$-Matrix mit $m \leq n$ und $b \in \R^m$.
\end{definition}

Das Problem~\eqref{prob:quad_opt_prob_mit_lin_gl_nebenbed} werden wir auf ein
unrestringiertes Verfahren reduzieren mit Hilfe einer Nullraum-Matrix von~$A$.
Daher kommt der Name \emph{Nullraum-Verfahren}.

\begin{algorithm}
\emph{(Nullraum-Verfahren)}
\begin{enumerate}%TODO 'mithilfe' oder 'mit Hilfe'
  \item Finde mit Hilfe der QR-Zerlegung
  unit�re $(n \times n)$-Matrix~$H$
  und obere $(m \times m)$-Dreiecksmatrix~$R$ mit
  \begin{equation}
    H A^T = \left(\begin{array}{c} R \\ 0 \end{array}\right).
  \end{equation}
  \item Berechne
  \begin{equation}
    \left(\begin{array}{c} h_1 \\ h_2 \end{array}\right) := -Hq
    \quad \text{und} \quad
    B := H Q H^T
       = \left(\begin{array}{cc} B_{11} & B_{12} \\ B_{21} & B_{22}
         \end{array}\right),
  \end{equation}
  wobei $h_1 \in \R^m$ und $B_{11} \in \R^{m,m}$
  \item Berechne den Vektor $u_1$ als L�sung der Gleichung
  \begin{equation}
    R^T u_1 = b
  \end{equation}
  und den Vektor $u_2$ als L�sung der Gleichung
  \begin{equation}
    B_{22} u_2 = h_2 - B_{21} u_1
  \end{equation}
  \item Berechne $\xopt$
  \begin{equation}
    \xopt := H^T \left(\begin{array}{c} u_1 \\ u_2 \end{array}\right)
  \end{equation}
  \item Der Multiplikator $\lambda^*$ ist die L�sung der Gleichung
  \begin{equation}
    R \lambda^* = h_1 - B_{11} u_1 - B_{12} u_2
  \end{equation}
\end{enumerate}
\end{algorithm}
