Zun�chst werden in diesem Unterkapitel
Optimierungsprobleme mit linearen Gleichungs- und
Ungleichungsnebenbedingungen~\eqref{prob:opt_prob_mit_lin_ungl_nebenbed}
betrachtet.
\begin{align*}
  \min_{x \in \R^n}\ & f(x) \\
                 \nb & Ax = b \\
                     & Gx \leq r
\end{align*}

Das Teilproblem nach~\eqref{prob:teilprob_von_sqp_allg} lautet dann
\begin{align}
  \min_{x \in \R^n}\ & \frac{1}{2} (x-x^k)^T f''(x^k) (x-x^k)
                         + \nabla f(x^k)^T (x-x^k) \\
                 \nb & A x = b \notag \\
                     & G x \leq r \notag.
\end{align}

Setze $d := x - x^k$.
Folglich ist $x = x^k + d$ und das Teilproblem l�sst sich schreiben als
\begin{align}
  \min_{x \in \R^n}\ & \frac{1}{2} d^T f''(x^k) d
                         + \nabla f(x^k)^T d \\
                 \nb & A x^k + A d = b \label{eq:teilprob_lin_gl_nebenbed} \\
                     & G x^k + G d \leq r
\end{align}

Das ist ein quadratisches Optimierungsproblem mit linearen Restriktionen.
Im n�chsten Unterkapitel wird ein Verfahren f�r dieses Problem vorgestellt.

Da $x^k \in \F$ ist, d.\,h., es gilt $A x^k = b$,
kann als Gleichungsnebenbedingung an
der Stelle~\eqref{eq:teilprob_lin_gl_nebenbed} die Gleichung $Ad = 0$
geschrieben werden.

\begin{algorithm}
\label{algo:sqp_verfahren_fuer_prob_mit_lin_ungl_nebenbed}
\emph{(SQP-Verfahren f�r \eqref{prob:opt_prob_mit_lin_ungl_nebenbed},
vgl. Verfahren 6.4.1 in \cite[S.~237]{alt})}
\begin{enumerate}
  \item W�hle einen zul�ssigen Startpunkt $x^0$ und setze $k:=0$.
  \item Berechne die L�sung $d$ des Problems \label{list:quad_opt_prob_in_sqp}
  \begin{align}
    \min_{x \in \R^n}\ & \frac{1}{2} d^T f''(x^k) d
                       + \nabla f(x^k)^T d
                       \tag{$\text{QLU}_k$}
                       \label{prob:quad_opt_prob_in_sqp}\\
                   \nb & A d = 0 \notag \\
                       & G x^k + G d \leq r \notag
  \end{align}
    Setze $d^k := d$.
    
  \item Ist $d^k = 0$ $\Rightarrow$ STOP.
    
  \item Setze $x^{k+1} := x^k + d^k$ und $k := k+1$.
    $\Rightarrow$ Gehe zu Schritt~\ref{list:quad_opt_prob_in_sqp}.
\end{enumerate}
\end{algorithm}

Es ist zu bemerken, dass
der Punkt $x^k$ die notwendige Optimalit�tsbedingung
des Problems~\eqref{prob:opt_prob_mit_lin_ungl_nebenbed}
nach Satz~\ref{satz:karush_kuhn_tucker} erf�llt,
falls $d^k = 0$ gilt.

Sei $d$ eine L�sung des
Teilproblems~\eqref{prob:quad_opt_prob_in_sqp}.
Dann gibt es nach Satz~\ref{satz:karush_kuhn_tucker}
die Vektoren $\lambda$ und $\mu$, sodass
\begin{gather}
  f''(x^k)d + \nabla f(x^k) + A^T \lambda + G^T \mu = 0 \\
  \mu_j \geq 0, \quad
  \mu_j (\langle g_j, x^k \rangle + \langle g_j, d \rangle - r_j) = 0,
    \quad j = 1,\ldots,p.
\end{gather}
Falls $d = 0$ ist, gilt
\begin{gather}
  \nabla f(x^k) + A^T \lambda + G^T \mu = 0 \\
  \mu_j \geq 0, \quad
  \mu_j (\langle g_j, x^k \rangle - r_j) = 0,
    \quad j = 1,\ldots,p,
\end{gather}
d.\,h., der Punkt $x^k$ erf�llt
die notwendige Optimalit�tsbedingung.

Der folgende Satz macht eine Aussage zur Konvergenzgeschwindigkeit
des Verfahrens.
\begin{theorem}
\label{satz:konvergenz_seq_quad_prog}
\emph{(Vgl. Satz 6.4.4 in \cite[S.~243]{alt})}\\
Sei $\xopt$ lokale L�sung von \eqref{prob:opt_prob_mit_lin_ungl_nebenbed}.
Sei $f$ auf $D$ zweimal stetig differenzierbar
und es gelte die Bedingung~\eqref{eq:hesse_matrix_pos_def_auf_kern_A}.
Dann gibt es ein $\delta > 0$, sodass das
Verfahren~\ref{algo:sqp_verfahren_fuer_prob_mit_lin_ungl_nebenbed}
f�r jeden Startpunkt $x^0 \in U_\delta(\xopt)$
eine Folge $\{x^k\}$ definiert,
die superlinear gegen $\xopt$ konvergiert, d.\,h.,
es existiert eine Nullfolge $\{r_k\} \subset \R_+$ mit
\begin{equation}
  \| x^{k+1} - \xopt \|
    \leq r_k \| x^k - \xopt \|
    \qquad \forall k \geq 0.
\end{equation} 
Ist zus�tzlich $f''$ auf $D$ Lipschitz-stetig\footnote{$f''$ ist
auf $D$ Lipschitz-stetig,
falls es eine Konstante $L > 0$ mit
$\|f''(x) - f''(y)\| \leq L \|x-y\|$
f�r alle $x,y \in D$
gibt.},
dann konvergiert die Folge $\{x^k\}$ f�r jeden Startpunkt
$x^0 \in U_\delta(\xopt)$ quadratisch gegen $\xopt$,
d.\,h., es existieren $c > 0$ und $K \in \N$ mit
\begin{equation}
  \| x^{k+1} - \xopt \|
    \leq c \| x^k - \xopt \|^2
    \qquad \forall k \geq K.
\end{equation} 
\end{theorem}

Es kann gezeigt werden, dass mit den Voraussetzungen des Satzes
die L�sbarkeit des Problems~\eqref{prob:quad_opt_prob_in_sqp}
gesichert werden kann.
Der Satz zeigt, dass das
Verfahren~\ref{algo:sqp_verfahren_fuer_prob_mit_lin_ungl_nebenbed}
leider nur lokal konvergiert.
Um ein global konvergentes Verfahren zu bekommen,
muss eine geeignete Schrittweite gew�hlt werden.