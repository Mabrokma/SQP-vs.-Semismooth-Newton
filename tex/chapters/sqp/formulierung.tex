Wir betrachen erstmal das Optimierungsproblem mit linearen
Gleichungs- und
Ungleichungsnebenbedingungen~\eqref{prob:opt_prob_mit_lin_ungl_nebenbed}.
\begin{align*}
  \min_{x \in \R^n}\ & f(x) \\
                 \nb & Ax = b \\
                     & Gx \leq r
\end{align*}

Das Teilproblem ist also
\begin{align}
  \min_{x \in \R^n}\ & \frac{1}{2} (x-x^k)^T f''(x^k) (x-x^k)
                         + \nabla f(x^k)^T (x-x^k) \\
                 \nb & A x = b \notag \\
                     & G x \leq r \notag
\end{align}%TODO erw�hnen dass f(x^k) weggelassen wurde?

Definieren wir $d := x - x^k$.
Dann haben wir $x = x^k + d$ und als Teilproblem
\begin{align}
  \min_{x \in \R^n}\ & \frac{1}{2} d^T f''(x^k) d
                         + \nabla f(x^k)^T d \\
                 \nb & A x^k + A d = b \label{eq:teilprob_lin_gl_nebenbed} \\
                     & G x^k + G d \leq r
\end{align}

Das ist ein quadratisches Optimierungsproblem mit linearen Restriktionen,
welches wir gleich n�her betrachten werden.
Weil wir voraussetzen werden, dass $x^k \in \F$ ist,
d.\,h., es gilt \mbox{$A x^k = b$}, k�nnen wir als Gleichungsnebenbedingung an
der Stelle~\eqref{eq:teilprob_lin_gl_nebenbed} die Gleichung $Ad = 0$ schreiben.
Nun sind wir bereit, das SQP-Verfahren zu formulieren.

\begin{algorithm}
\emph{(SQP-Verfahren)}
\begin{enumerate}
  \item W�hle einen zul�ssigen Startpunkt $x^0$
        und ein Abbruchkriterium $\epsilon > 0$. Setze $k:=0$.
  \item Berechne die L�sung $d$ des Problems \label{list:quad_opt_prob}
  \begin{align}
    \min_{x \in \R^n}\ & \frac{1}{2} d^T f''(x^k) d
                       + \nabla f(x^k)^T d \label{prob:quad_opt_prob_in_sqp}\\
                   \nb & A d = 0 \notag \\
                       & G x^k + G d \leq r. \notag
  \end{align}
        Setze $d^k := d$.
  \item Ist $||d^k|| < \epsilon \Rightarrow$ STOP.
  \item Setze $x^{k+1} := x^k + d^k$ und $k := k+1$.
        $\Rightarrow$ Gehe zu Schritt~\ref{list:quad_opt_prob}.
\end{enumerate}
\end{algorithm}
