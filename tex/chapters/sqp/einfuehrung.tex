Das SQP-Verfahren ist ein wichtiges Verfahren,
um restringierte nichlineare Optimierungsprobleme zu l�sen.
SQP ist eine Abk�rzung f�r sequentielle quadratische
Programmierung\footnote{englisch: Sequential Quadratic Programming}.
Die Hauptidee von SQP ist n�mlich, iterativ Teilprobleme in Form quadratischer
Optimierungsprobleme zu formulieren und zu l�sen.

Wir haben im Unterkapitel~\ref{sec:newton_verfahren} �ber das Newton-Verfahren
gesehen, dass dieses Verfahren in jedem Iterationschritt eigentlich ein
unrestringiertes quadratisches Problem
\begin{equation}
  \min_{x \in \R^n}\ \frac{1}{2} (x-x^k)^T f''(x^k) (x-x^k)
                     + \nabla f(x^k)^T (x-x^k)
\end{equation}
l�st.
Diese Idee wird bei SQP auch verwendet. Aber es wird nicht mehr ein
unrestringiertes Problem sein, weil die Nebenbedingungen auch
ber�cksichtigt werden, d.\,h.
\begin{equation} \label{prob:teilprob_von_sqp_allg}
  \min_{x \in \F}\ \frac{1}{2} (x-x^k)^T f''(x^k) (x-x^k)
                     + \nabla f(x^k)^T (x-x^k).
\end{equation}
