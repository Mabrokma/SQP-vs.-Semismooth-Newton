Das SQP-Verfahren ist ein wichtiges Verfahren,
um restringierte nichlineare Probleme zu l�sen.
SQP ist eine Abk�rzung f�r sequentielle quadratische Programmierung
(auf Englisch: Sequential Quadratic Programming).
Die Idee ist n�mlich, dass Teilprobleme in Form quadratischer
Optimierungsprobleme iterativ formuliert und gel�st werden.

%Erstes: Wilson im Jahr 1963
% [120] Wilson R. B.: A simplical algorithm for concave programming. PhD Thesis.
% Harvard university, Boston 1963.
% [92] Han und Powell: Algorithms 
% [41]

\begin{align}
  \min_{d \in \R^n}\ & \frac{1}{2} d^T f''(x^k) d + \nabla f(x^k)^T d
    \label{prob:quad_opt_prob_in_sqp} \\
                 \nb & A d = 0, \notag \\
                    & G x^k + G d \leq r \notag
\end{align}

\begin{algorithm}
\emph{(SQP-Verfahren)}
\begin{enumerate}
  \item Berechne einen zul�ssigen Startpunkt $x^0$ und setze $k:=0$.
  \item Berechne die L�sung $d$ des Problems~\eqref{prob:quad_opt_prob_in_sqp}.
        \label{list:quad_opt_prob} Setze $d^k := d$.
  \item Setze $x^{k+1} := x^k + d^k$ und $k := k+1$.
        $\Rightarrow$ Gehe zu Schritt~\ref{list:quad_opt_prob}
\end{enumerate}
\end{algorithm}

