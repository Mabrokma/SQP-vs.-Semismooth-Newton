
Wir haben gesehen, dass das SQP-Verfahren
quadratische Optimierungsprobleme mit linearen Restriktionen
als Teilprobleme hat.
Wir werden jetzt aufkl�ren, wie wir diese Probleme l�sen werden.

\begin{definition}
\emph{(Quadratische Optimierungsprobleme mit linearen Restriktionen)}
\begin{align}
  \min_{x \in \R^n}\ &
    \frac{1}{2} \langle Qx,x \rangle + \langle q,x \rangle 
    \tag{QLG} \label{prob:quad_prog_mit_lin_restr} \\
  \nb & Ax = b \notag \\
      & Gx \leq r \notag
\end{align}
Dabei sei $Q$ eine symmetrische $(n \times n)$-Matrix und $q \in \R^n$.
$A$ sei eine $(m \times n)$-Matrix mit $m \leq n$ und $b \in \R^m$.
$G$ sei eine $(p \times n)$-Matrix und $r \in \R^p$.
\end{definition}

Sei $J(x)$ die Indexmenge der aktiven Ungleichungsrestriktionen.
$B_k := \left(\begin{array}{c} A \\ G(x) \end{array}\right)$

\begin{align}
  \min_{d \in \R^n}\ & \frac{1}{2} \langle Qd,d \rangle
                       + \langle Qx^k+q,d\rangle
                         \label{prob:teil_prob_in_akt_me_verf} \\
                 \nb & B_k d = 0 \notag
\end{align}

\begin{algorithm}
\emph{(Aktive-Mengen-Methode f�r~\eqref{prob:quad_prog_mit_lin_restr})}
\begin{enumerate}
  \item W�hle einen Startpunkt $x^0$. Setze $k:=0$.
  \item L�se das Problem~\eqref{prob:teil_prob_in_akt_me_verf}
\end{enumerate}
\end{algorithm}