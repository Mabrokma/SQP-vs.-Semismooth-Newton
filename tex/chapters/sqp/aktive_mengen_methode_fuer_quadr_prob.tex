Es ist jetzt aufzukl�ren,
wie die im SQP-Verfahren vorkommenden quadratischen Teilprobleme
mit dem Aktive-Mengen-Verfahren
gel�st werden k�nnen.

%-------------------------------------------------------------------------%
%                                                                         %
% Definition (Quadr. Opt.-Prob. mit lin. Ungl.-Nebenbed.)                 %
%                                                                         %
%-------------------------------------------------------------------------%

\begin{definition}
\emph{(Quadratische Optimierungsprobleme mit linearen Restriktionen)}
\begin{align}
  \min_{x \in \R^n}\ & f(x) :=
    \frac{1}{2} \langle Qx,x \rangle + \langle q,x \rangle 
    \tag{QLU} \label{prob:quad_opt_prob_mit_lin_ungl_nebenbed} \\
                 \nb & Ax = b \notag \\
                     & Gx \leq r \notag
\end{align}
Dabei sei $Q$ eine symmetrische $(n \times n)$-Matrix und $q \in \R^n$.
$A$ sei eine $(m \times n)$-Matrix mit $m \leq n$ und $b \in \R^m$.
$G$ sei eine $(p \times n)$-Matrix und $r \in \R^p$.
\end{definition}

Die Idee des Aktive-Mengen-Verfahrens ist,
iterativ durch die aktiven Nebenbedingungen
ein Problem mit nur Gleichungsnebenbedingungen zu formulieren
und zu l�sen.

Sei $x^k \in \R^n$ ein zul�ssiger Punkt des
Problems~\eqref{prob:quad_opt_prob_mit_lin_ungl_nebenbed}.
Angenommen, eine Suchrichtung $d \in \R^n$ kann gefunden werden,
sodass der Punkt $x^{k+1} := x^k + d$ ein besserer zul�ssiger Punkt ist.
Diesen Vektor $d$ l�sst sich finden, indem das Problem
\begin{align}
  \min_{d \in \R^n}\ &
    \frac{1}{2} \langle Q(x^k+d), x^k+d \rangle + \langle q, x^k+d \rangle 
      \\
                 \nb & A(x^k+d) = b \notag \\
                     & G(x^k+d) \leq r \notag
\end{align}
gel�st wird.
Dies ist eigentlich das Problem~\eqref{prob:quad_opt_prob_mit_lin_ungl_nebenbed},
wobei die Variable $x$ mit $x^k + d$ ersetzt wurde und das Problem �ber
die Variable $d$ minimiert werden soll.

Da die Konstanten in der Zielfunktion vernachl�ssigt werden k�nnen,
ergibt sich als Zielfunktion
\begin{displaymath}
  \frac{1}{2} \langle Qd,d \rangle + \langle Qx^k+q,d\rangle.
\end{displaymath}
Weil $x^k$ zul�ssig ist, d.\,h., es gilt $Ax^k = b$,
vereinfacht sich die Gleichungsnebenbedingung als
$Ad = 0$.

Seien $a_1,\ldots,a_m \in \R^n$ bzw. $g_1,\ldots,g_p \in \R^n$
wieder die Transponierten der Zeilenvektoren in $A$ bzw. $G$
wie in~\eqref{eq:zerlegung_von_A_und_G} und
$J(x^k) := \{ 1 \leq j \leq p \ | \ \langle g_j,x^k \rangle = r_j \}$
die Indexmenge der aktiven Ungleichungsrestriktionen zu $x^k$
wie in~\eqref{eq:indexmenge_der_aktiven_restr}.

Ist eine Ungleichungsrestriktion zu $x^k$ mit Index $j$ aktiv,
dann kann diese Ungleichungsrestriktion auch zu $x^k+d$ aktiv bleiben,
wenn $\langle g_j, d \rangle = 0$ gilt.

Seien $J_k := J(x^k)$, $p_k := |J_k|$ und
$j_1, \ldots, j_{p_k} \in J_k$.
Weiter seien
\begin{equation}
  G_k = \left(\begin{array}{c}
      g_{j_1}^T \\ \vdots \\ g_{j_{p_k}}^T
    \end{array}\right)
  \quad \text{und} \quad
  B_k = \left(\begin{array}{c}
      A \\ G_k
    \end{array}\right).
\end{equation}

Definiere dann folgendes Teilproblem mit nur
linearen Gleichungsnebenbedingungen:
\begin{align}
  \min_{d \in \R^n}\ & \frac{1}{2} \langle Qd,d \rangle
                       + \langle Qx^k+q,d\rangle
    \tag{$\text{QLG}_k$} \label{prob:teilprob_in_aktive_mengen_verfahren} \\
  \nb & B_k d = 0. \notag
\end{align}

Es ist hier vorauszusetzen, dass folgende gelten:
\begin{gather}
  \label{eq:vorauss_Bk_hat_voll_rang}
  B_k \text{ hat vollen Rang},\\
  \text{d.\,h., } a_1,\ldots,a_m,\ g_{j_1},\ldots,g_{j_{p_k}}
  \text{ sind linear unabh�ngig,} \nonumber
\end{gather}
und
\begin{equation}
  \label{eq:vorauss_Q_pos_def_auf_ker_Bk}
  Q \text{ ist positiv definit auf } \kr{B_k}.
\end{equation}

Sei $d$ eine L�sung des
Teilproblems~\eqref{prob:teilprob_in_aktive_mengen_verfahren}.
Dann existiert nach Satz~\ref{satz:notw_bed_fuer_prob_mit_lin_gl_nebenbed}
ein Lagrange-Multiplikator
$\nu = \left(\begin{array}{c} \lambda \\ \mu \end{array}\right)$
mit $\lambda \in \R^m$ und $\mu \in \R^{p_k}$, sodass
\begin{equation}
\label{eq:notw_bed_teilprob_in_aktive_mengen_verfahren}
Qd + Qx^k + q + A^T \lambda + G_k^T \mu = 0.
\end{equation}

Sei $d \neq 0$.
Dann ist $d$ eine Abstiegsrichtung
(siehe Definition~\ref{def:abstiegsrichtung}).

F�r den Gradient der Zielfunktion des
Problems~\eqref{prob:quad_opt_prob_mit_lin_ungl_nebenbed}
an der Stelle $x^k$ gilt
wegen~\eqref{eq:notw_bed_teilprob_in_aktive_mengen_verfahren}
\begin{equation}
\nabla f(x^k) = Qx^k + q = -Qd - A^T \lambda - G_k^T \mu
\end{equation}
und somit
\begin{equation}
\nabla f(x^k)^T d = -d^T Q d - \lambda^T A d - \mu^T G_k d
  = -d^T Q d,
\end{equation}
weil $Ad=0$ und $G_k d=0$.
Wegen der Bedingung~\eqref{eq:vorauss_Q_pos_def_auf_ker_Bk}
gilt $d^T Q d > 0$ und folglich
\begin{equation}
  \nabla f(x^k)^T d < 0,
\end{equation}
d.\,h., $d$ ist eine Abstiegsrichtung.

Daher kann nach Satz~\ref{satz:existenz_von_schrittweite}
eine Schrittweite $\sigma$ berechnet werden, sodass
\begin{equation}
  f(x^{k+1} := x^k + \sigma d) < f(x^k)
\end{equation}
gilt.

Sei $t$ eine Schrittweite. Es gilt
\begin{align}
f(x^k+td) & = \tfrac{1}{2} (x^k+td)^T Q (x^k+td) + q^T (x^k+td) \\
  & = \tfrac{1}{2} (x^k)^T Q x^k + t (x^k)^T Q d + \tfrac{1}{2} t^2 d^T Q d
    + q^T x^k + t q^T d \\
  & = f(x^k) + t (x^k)^T Q d + \tfrac{1}{2} t^2 d^T Q d + t q^T d \\
  & = f(x^k) + \tfrac{1}{2} t^2 d^T Q d + t (Q x^k + q)^T d \\
  & = f(x^k) + \tfrac{1}{2} t^2 d^T Q d + t \nabla f(x^k)^T d \\
  & = f(x^k) + \tfrac{1}{2} t^2 d^T Q d - t d^T Q d \\
  & = f(x^k) + (\tfrac{1}{2} t^2 - t) d^T Q d
\end{align}

Die Funktion $\tfrac{1}{2} t^2 - t$ hat ein globales Minimum
an der Stelle $t = 1$ mit dem Optimalwert $-\tfrac{1}{2}$.
Man erh�lt also mit $t = 1$ den gr��ten Abstieg.

Es ist noch zu beachten,
dass $x^{k+1} := x^k + td$
nicht unbedingt ein zul�ssiger Punkt ist.
Durch die Bedingungen $Ad = 0$ und $G_k d = 0$
bekommt man zwar
\begin{equation}
  A x^{k+1} = A x^k + tAd = A x^k \leq 0
\end{equation}
und
\begin{equation}
\langle g_j, x^{k+1} \rangle
  = \langle g_j, x^k \rangle + t \langle g_j, d \rangle
  = \langle g_j, x^k \rangle
  = r_j \text{ f�r } j \in J_k,
\end{equation}
aber es kann sein,
dass eine Ungleichung
\begin{equation}
\langle g_j, x^{k+1} \rangle \leq r_j
\text{ f�r } j \in \{ 1, \ldots, p \} \backslash J_k
\end{equation}
verletzt wurde.

Sei nun $j \in \{ 1, \ldots, p \} \backslash J_k$.
Falls $\langle g_j, d \rangle < 0$ gilt,
dann erf�llt $x^{k+1}$
\begin{equation}
\langle g_j, x^{k+1} \rangle
  = \langle g_j, x^k \rangle + t \langle g_j, d \rangle
  < \langle g_j, x^k \rangle
  \leq r_j.
\end{equation}
F�r den Fall $\langle g_j, d \rangle > 0$ gilt
\begin{equation}
\langle g_j, x^{k+1} \rangle
  = \langle g_j, x^k \rangle + t \langle g_j, d \rangle
  \leq r_j
\end{equation}
nur dann, wenn $t$ die Ungleichung
\begin{equation}
\label{eq:schrittweite_bedingung_in_aktive_mengen_verfahren}
t \leq \frac{r_j - \langle g_j, x^k \rangle}{\langle g_j, d \rangle}
\end{equation}
erf�llt.
Sei
\begin{equation}
I_k := \{\ j \in \{ 1, \ldots, p \} \backslash J_k
\ | \ \langle g_j,d \rangle > 0 \ \}.
\end{equation}
Dann muss $t$ die
Ungleichung~\eqref{eq:schrittweite_bedingung_in_aktive_mengen_verfahren}
f�r alle $j \in I_k$ erf�llen.

Zusammengefasst ist die Schrittweite $\sigma_k$
wie folgt zu definieren.
\begin{equation}
\label{eq:schrittweite_formel_fuer_aktive_mengen_verf}
\sigma_k := \min \{ 1, \tau_k\},
\end{equation}
wobei
\begin{equation}
\tau_k :=
  \begin{cases}
    \min\left\{ \frac{r_j - \langle g_j, x^k \rangle}{\langle g_j, d \rangle}
      \ | \ j \in I_k \right\}, & \text{falls } I_k \neq \emptyset \\
    \infty,
      & \text{sonst}.
  \end{cases}
\end{equation}

Es ist nun der Fall $d = 0$ zu betrachten.
In diesem Fall lautet die
Gleichung~\eqref{eq:notw_bed_teilprob_in_aktive_mengen_verfahren}
\begin{equation}
\label{eq:notw_bed_teilprob_in_aktive_mengen_verf_mit_d_gl_null}
  Qx^k + q + A^T \lambda + G_k^T \mu = 0.
\end{equation}

Sei $\mu \geq 0$.
Setze
\begin{equation}
  \lambda^* := \lambda
  \quad \text{und} \quad
  \mu^* = (\mu_j^*)_{j=1,\ldots,p} :=
    \begin{cases}
      \mu_j, & \text{falls } j \in J_k \\
      0, & \text{sonst}.
    \end{cases}
\end{equation}
Dann erf�llt $x_k$ die notwendigen
Bedingungen von Satz~\ref{satz:karush_kuhn_tucker}
\begin{align}
\nabla f(x^k) + A^T \lambda^* + G^T \mu^* &=
    Qx^k + q + A^T \lambda + G_k^T \mu= 0 \\
\mu_j^* (\langle g_j, x^k \rangle - r_j ) & = 0 \text{ f�r } j = 1,\ldots,p \\
                                    \mu^* & \geq 0
\end{align}
und wegen Voraussetzung~\eqref{eq:vorauss_Q_pos_def_auf_ker_Bk}
auch die hinreichende
Optimalit�tsbedingung~\eqref{satz:hinr_bed_fuer_prob_mit_lin_ungl_nebenbed}.
Daher ist $x^k$ eine optimale L�sung
von~\eqref{prob:quad_opt_prob_mit_lin_ungl_nebenbed}
und das Verfahren kann somit gestoppt werden.

Falls $\mu \ngeq 0$, dann w�hle den Index $j \in J_k$ mit
\begin{equation}
  \mu_j = \min\{ \mu_i | i \in J_k \} < 0
\end{equation}
und setze
\begin{equation}
  \tilde{J}_k := J_k \backslash \{\,j\}
  \quad \text{und} \quad
  \tilde{p}_k := p_k - 1.
\end{equation}
Mit $\tilde{J}_k$ bekommt man eine neue Matrix $\tilde{B}_k$
und ein neues Problem~\eqref{prob:teilprob_in_aktive_mengen_verfahren}.
Die $j$-te Ungleichung wird somit nicht mehr als aktiv angesehen.

Sei $\tilde{d}$ die L�sung des neuen Problems mit den zugeh�rigen
Multiplikatoren $\tilde{\lambda}$ und $\tilde{\mu}$.
Es gilt die notwendige Bedingung
\begin{equation}
Q\tilde{d} + Qx^k + q + A^T \tilde{\lambda} + G_k^T \tilde{\mu} = 0.
\end{equation}

Angenommen ist $\tilde{d} = 0$. Dann lautet die notwendige Bedingung
\begin{equation}
\label{eq:notw_bed_fuer_QLGk_tilde}
  Qx^k + q + A^T \tilde{\lambda} + G_k^T \tilde{\mu} = 0.
\end{equation}
Fasst man diese mit der
Gleichung~\eqref{eq:notw_bed_teilprob_in_aktive_mengen_verf_mit_d_gl_null}
zusammen,
dann bekommt man
\begin{equation}
-A^T \lambda - G_k^T \mu + A^T \tilde{\lambda} + \tilde{G_k}^T \tilde{\mu} = 0
\end{equation}
\begin{equation}
\sum_{i=1}^{m} (\tilde{\lambda}_i - \lambda_i) a_i
+ \sum_{i=1}^{p_k-1} (\tilde{\mu}_i - \mu_i) g_{j_i}
= \mu_j g_j
\end{equation}
F�r die letzte Gleichung wird o.\,B.\,d.\,A. angenommen, dass $j = p_k$.

Wegen $\mu_j < 0$ hei�t es, dass $g_j$ durch die Linearkombination
der Vektoren $a_1,\ldots,a_m$ und $g_{j_1},\ldots,g_{j_{\tilde{p}_k}}$
darstellbar ist.
Das ist ein Widerspruch zu der
Voraussetzung~\eqref{eq:vorauss_Bk_hat_voll_rang}.
Daher gilt $\tilde{d} \neq 0$
und dies w�re wie zuvor beschrieben eine Abstiegsrichtung.

%--------------------------------------------------------------------------%
%                                                                          %
% Algo (Aktive-Mengen-Verf. f. quadr. Opt.-Prob. mit lin. Ungl.-Nebenbed.) %
%                                                                          %
%--------------------------------------------------------------------------%

\begin{algorithm}
\label{algo:aktive_mengen_verfahren}
\emph{(Aktive-Mengen-Verfahren
f�r \eqref{prob:quad_opt_prob_mit_lin_ungl_nebenbed},
vgl. Verfahren 6.2.1 in \cite[S.~213]{alt})}
\begin{compactenum}[1.]
  \item W�hle einen Startpunkt $x^0$ und setze $k:=0$.
  \item L�se das Problem~\eqref{prob:teilprob_in_aktive_mengen_verfahren}
  	$\Rightarrow$ Erhalte $d^k, \lambda^k$ und $\mu^k$.
  	\label{list:quad_teilprob_in_aktive_mengen_verf}
  \item Falls $d^k = 0$
    \begin{compactenum}[3a.]
      \item Falls $\mu^k \geq 0 \Rightarrow$ STOP.
      \item Falls $\mu^k \ngeq 0$
        \begin{compactenum}[i.]
          \item Bestimme $j \in J_k$, sodass
            $\mu_j^k = \min\{ \mu_i^k \ |\ i \in J_k \}$.
          \item Setze $\tilde{J}_k := J_k \backslash\{\,j\}$
            und bestimme $\tilde{G}_k$.
          \item L�se das neue
            Problem~\eqref{prob:teilprob_in_aktive_mengen_verfahren}
            $\Rightarrow$ Erhalte $d^k \neq 0, \lambda^k$ und $\mu^k$.
        \end{compactenum}
    \end{compactenum}
  \item Berechne die Schrittweite $\sigma_k$
    nach~\eqref{eq:schrittweite_formel_fuer_aktive_mengen_verf}.
  \item Setze $x^{k+1} := x^k + \sigma_k d^k$ und $k := k+1$.
    $\Rightarrow $ Gehe zu Schritt
  \ref{list:quad_teilprob_in_aktive_mengen_verf}.
\end{compactenum}
\end{algorithm}

Es kann gezeigt werden, dass das Verfahren~\ref{algo:aktive_mengen_verfahren}
in endlich vielen Schritten die L�sung des
Problems~\eqref{prob:quad_opt_prob_mit_lin_ungl_nebenbed}
berechnet (siehe Satz 6.2.3 in \cite[S.~214]{alt}).