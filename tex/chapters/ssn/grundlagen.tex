Die Definition einer halbglatten Funktion
kommt vorerst in Betrachtung.

\begin{definition}
\emph{(Vgl. Definition 2.1 in \cite[S.~25]{ulbrich})}\\
Sei $V \subset \R^n$ eine nichtleere offene Menge.
Sei $f \colon V \rightarrow \R^m$ Lipschitz-stetig in der N�he von $x \in V$,
d.\,h., es gibt eine offene Menge $V(x) \subset V$, wo $f$ Lipschitz-stetig ist.
Sei $D_f \subset V$ die Menge aller Punkte $v \in V$, wo $f$ eine Ableitung
%\footnote{(Fr\'echet-)Ableitung einer Abbildung $F\colon \R^n \to \R^m$ an der
% Stelle $x \in \R^n$ ist $F'(x) \in L(\R^n,\R^m)$ mit
%\[ \norm{F(x+s) - F(x) - F'(x)s} = o(\norm{s}) \text{ f�r } \norm{s} \to 0.\]}
$f'(v) \in \R^{m,n}$ besitzt.
Die Menge
\begin{equation}
  \partial_B f(x) := \left\{ M \in \R^{m \times n}\ |\ 
    \exists\, \{x_k\} \subset D_f :
    x_k \rightarrow x,\: f'(x_k) \rightarrow M \right\}
\end{equation}
hei�t B-Sub-Ableitung\footnote{englisch: B-subdifferential,
\textss{B} f�r Bouligand.}
von $f$ an der Stelle $x$.
Ferner ist nach Clarke die allgemeine Jacobi-Matrix
von $f$ an der Stelle $x$
die konvexe H�lle\footnote{Die konvexe Hulle einer Menge $M$ ist
\[\textstyle \co{M} := \left\{ \sum_{i=1}^k \lambda_i v_i \left|\:
k \in \N,\: v_i \in M,\: \lambda_i \geq 0 \text{ f�r } i=1,\ldots,k,\: 
\sum_{i=1}^k \lambda_i = 1\: \right\}. \right. \]}
\begin{equation}
  \partial f(x) := \co \partial_B f(x).
\end{equation}
\end{definition}

\begin{definition}
\emph{(Vgl. Definition 2.5 in \cite[S.~27]{ulbrich})}\\
Sei $V \subset \R^n$ nichtleer und offen.
Die Funktion $f \colon V \rightarrow \R^m$ hei�t halbglatt in $x \in V$,
falls $f$ lokal Lipschitz-stetig in $x$ ist und
\begin{equation}
  \lim_{M \in \partial f(x+ \tau d),\: d \rightarrow s,\: \tau \rightarrow 0^+}
    Md
\end{equation}
f�r alle $s \in \R^m$ existiert.

Falls $f$ halbglatt in allen $x \in V$ ist,
dann hei�t $f$ halbglatt (in $V$).  
\end{definition}

\begin{example}
\emph{(Vgl. Abschnitt 2.5.1 in \cite[S.~31~f.]{ulbrich})}\\
Die Funktion
\begin{equation}
  f \colon \R^n \to \R,\: x \mapsto \norm{x} = \sqrt{x^T x}
\end{equation}
ist eine halbglatte Funktion mit
\begin{equation}
  \partial f(x) = \partial_B f(x) =
    \left\{ \frac{x^T}{\norm{x}} \right\} \text{ f�r } x \neq 0,
\end{equation}
\begin{equation}
  \partial_B f(0) = \left\{ v^T |\, v \in \R^n, \norm{v} = 1 \right\}
  \quad \text{und} \quad
  \partial f(0) = \left\{ v^T |\, v \in \R^n, \norm{v} \leq 1 \right\}.
\end{equation}
\end{example}

\begin{example}
\emph{(Vgl. Abschnitt 2.5.3 in \cite[S.~32~ff.]{ulbrich})}\\
St�ckweise differenzierbare Funktionen
sind halbglatt.
Ein Beispiel ist die Funktion
\begin{equation}
  f \colon \R^2 \to \R,\: x \mapsto \min\{ x_1, x_2 \}.
\end{equation}
\end{example}

Jetzt kommt die Formulierung des halbglatten Newton-Verfahrens
zum L�sen der Gleichung
\begin{equation}
  f(x) = 0, \label{eq:halbglatte_gleichung}
\end{equation}
wobei
$f \colon V \rightarrow \R^n$, $V \subset \R^n$ offen,
halbglatt an der Stelle der L�sung $\xopt \in V$ ist.

\begin{algorithm}\label{algo:halbglattes_newton}
\emph{(Das halbglatte Newton-Verfahren,
vgl. Algorithmus 2.11 in \cite[S.~29]{ulbrich})}
\begin{enumerate}
  \item W�hle einen Startpunkt $x^0$ und setze $k := 0$.
  \item Ist $f(x^k) = 0$ \label{list:stop_criteria_ssn}
        $\Rightarrow$ STOP.
  \item W�hle $M_k \in \partial f (x^k)$.
  \item Berechne die L�sung $d$ des linearen Gleichungssystems
        \begin{equation}
          M_k d = - f(x^k).
        \end{equation}
        Setze $d^k := d$.
  \item Setze $x^{k+1} := x^k + d^k$ und $k := k+1$. $\Rightarrow$
        Gehe zu Schritt~\ref{list:stop_criteria_ssn}.
\end{enumerate}
\end{algorithm}

Das halbglatte Newton-Verfahren arbeitet prinzipiell
wie das Newton-Verfahren.
Der Unterschied liegt darin, dass hier die Funktion nicht
unbedingt glatt ist.

\begin{theorem}
\emph{(Vgl. Satz 2.12 in \cite[S.~29~f.]{ulbrich})}\\
Sei $f \colon V \rightarrow \R^n$ auf der offenen Menge $V \subset \R^n$
definiert.
Sei $\xopt$ eine L�sung des Problems~\eqref{eq:halbglatte_gleichung}.
Es gelte:
\begin{compactenum}[(i)]
  \item Die Funktion $f$ sei halbglatt in $\xopt$.
  \item Es existiere eine Konstante $C > 0$, sodass die Matrizen $M_k$
    f�r alle $k$ nichtsingul�r mit $\| M_k^{-1} \| \leq C$ sind.
    \label{list:loesbarkeit_der_gleichung_in_halbgl_newton_verf}
\end{compactenum}
Dann existiert $\delta > 0$, sodass f�r alle
$x^0 \in U_{\delta}(\xopt)$
die Bedingung~(\ref{list:loesbarkeit_der_gleichung_in_halbgl_newton_verf})
erf�llt ist und das Verfahren~\ref{algo:halbglattes_newton}
entweder mit $x^k = \xopt$ terminiert
oder eine Folge $\{x^k\}$ erzeugt,
die superlinear gegen $\xopt$ konvergiert.
\end{theorem}
