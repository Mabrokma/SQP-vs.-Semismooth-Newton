Wir betrachten in diesem Unterkapitel
die Aktive-Menge-Strategie.
Dieses Verfahren ist f�r uns interessant,
weil sie als das halbglatte Newton-Verfahren interpretiert werden kann.

Wir betrachten wieder das Problem
\begin{align}
  \min_{x \in \R^n}\ & f(x)
    \tag{PLU} \label{} \\
  \nb & G x \leq r \notag
\end{align}

Eine der Optimalit�tsbedingungen dieses Problems ist:
\begin{equation}
  \nabla f(x) + G^T \mu = 0.
\end{equation}
Das ist ein nichtlineares Gleichungssystem, wenn $f$ nichtlinear ist.

Wir versuchen nun, eine linearisierte Version zu bekommen.

Sei $x^k$ ein zul�ssiger Punkt des Problems.
Approximieren wir die Funktion $f$ in der Umgebung von $x^k$:
\begin{equation}
  f(x) \approx f(x^k) + \nabla f(x^k)^T (x-x^k)
                 + \frac{1}{2} (x-x^k)^T f''(x^k) (x-x^k).
\end{equation}
Der Gradient von $f$ ist dann
\begin{equation}
  \nabla f(x) \approx \nabla f(x^k) + f''(x^k) x - f''(x^k) x^k.
\end{equation}

Ersetzen wir nun $\nabla f(x)$ in der Optimalit�tsbedingung mit seiner
Approximation, dann bekommen wir die Gleichung
\begin{align}
  \nabla f(x^k) + f''(x^k) x - f''(x^k) x^k + G^T \mu
    &= 0 \\
  f''(x^k) x + G^T \mu
    &= f''(x^k) x^k - \nabla f(x^k)
\end{align}

Das ist nun ein lineares Gleichungssystem.
Dieses werden wir in unserer Aktive-Mengen-Strategie verwenden.

\begin{algorithm}
\emph{(Aktive-Menge-Strategie)}
\begin{enumerate}
  \item W�hle $x^0$ und setze $k := 0$. 
  \item Bestimme
        \begin{align}
        \A_k & := \{i \in \{1,\ldots,p\}\ |\ r_i - g_i^T x^k < \mu_i \}, \\
        \I_k & := \{i \in \{1,\ldots,p\}\ |\ r_i - g_i^T x^k \geq \mu_i \}
        \end{align}
  \item L�se das Problem
        \begin{align}
          f''(x^k) x + G^T \mu
            &= f''(x^k) x^k - \nabla f(x^k) \label{eq:eq1_in_ass} \\
          g_i^T x &= r_i \quad \text{ f�r } i \in \A_k
            \label{eq:eq2_in_ass} \\
          \mu_i & = 0   \quad \text{ f�r } i \in \I_k
            \label{eq:eq3_in_ass}
        \end{align}
  \item Falls
        \begin{align}
          \nabla f(x) + \mu & = 0\\
          \min(\mu, r - G x) & = 0
        \end{align}
        $\Rightarrow$ STOP
  \item $x^{k+1} := x, \mu^{k+1} = \mu, k := k+1$
        $\Rightarrow$ Gehe zu Schritt 2. %TODO ref
\end{enumerate}
\end{algorithm}

Wir werden nun das Verfahren genauer betrachten und zeigen,
warum es als das halbglatte Newton-Verfahren interpretiert werden kann.

Sei $k$ irgendeiner Iterationsschritt in den beiden Verfahren.
Bei dem halbglatten Newton-Verfahren m�ssen wir dann
folgendes Gleichungssystem l�sen:
\begin{equation}
  \left(\begin{array}{cc}
         f''(x^k)         &       G^T \\
    - \chi_{\A_k}(1) g_1^T  &  \chi_{\I_k}(1) e_1^T \\
        \vdots          &      \vdots \\
    - \chi_{\A_k}(p) g_p^T  &  \chi_{\I_k}(p) e_p^T
  \end{array}\right)
  \left(\begin{array}{c} d_x \\ d_\mu \end{array}\right)
  = -
  \left(\begin{array}{c}
    \nabla f(x^k) + G^T \mu^k \\
    \chi_{\I_k}(1) \mu_1 + \chi_{\A_k}(1) (r_1 - g_1^T x^k) \\
      \vdots \\
    \chi_{\I_k}(p) \mu_p + \chi_{\A_k}(p) (r_p - g_p^T x^k)
  \end{array}\right)
\end{equation}

D.\,h.
\begin{align}
  f''(x^k) d_x + G^T d_\mu & = -\nabla f(x^k) - G^T \mu^k \\
  - \chi_{\A_k}(j) g_j^T d_x + \chi_{\I_k}(j) e_j^T d_\mu
    & = -\chi_{\I_k}(j) \mu_j - \chi_{\A_k}(j) (r_j - g_j^T x^k)
    \quad \text{f�r } j = 1,\ldots,p \label{eq:eq2in_ssn}
\end{align}

Die Indexmengen $\A_k$ und $\I_k$ sind dabei genau wie
in der Aktive-Mengen-Strategie.

Bei dem halbglatten Newton-Verfahren setzen wir dann
$x^{k+1} := x^k + d_x$ und $\mu^{k+1} := \mu^k + d_\mu$.
Es gilt also $d_x = x^{k+1} - x^k$ und $d_\mu = \mu^{k+1} - \mu^k$.
D.\,h.
\begin{align}
  f''(x^k) (x^{k+1} - x^k) + G^T (\mu^{k+1} - \mu^k)
    & = -\nabla f(x^k) - G^T \mu^k \\
  \Rightarrow \qquad f''(x^k) x^{k+1} + G^T \mu^{k+1} & =
    f''(x^k) x^k - \nabla f(x^k)
\end{align}
Das ist genau die Gleichung~\eqref{eq:eq1_in_ass}
in der Aktive-Mengen-Strategie.

Sei $j \in \A_k$.
Die Gleichung~\eqref{eq:eq2in_ssn} ist dann gleich
\begin{align}
  - g_j^T d_x & = - (r_j - g_j^T x^k) \\
  \Rightarrow - g_j^T (x^{k+1} - x^k) & = - (r_j - g_j^T x^k) \\
  \Rightarrow   g_j^T x^{k+1} & = r_j
\end{align}
Das ist genau die Gleichung~\eqref{eq:eq2_in_ass}
in der Aktive-Mengen-Strategie.

Sei nun $j \in \I_k$.
Die Gleichung~\eqref{eq:eq2in_ssn} ist dann gleich
\begin{align}
  e_j^T d_\mu & = - \mu_j \\
  \Rightarrow e_j^T (\mu^{k+1} - \mu^k) & = - \mu_j \\
  \Rightarrow \mu^{k+1}_j - \mu^k_j & = - \mu_j \\
  \Rightarrow \mu^{k+1}_j & = 0
\end{align}
Das ist genau die Gleichung~\eqref{eq:eq3_in_ass}
in der Aktive-Mengen-Strategie.

Somit sind die Iterationsschritte in den beiden Verfahren gleich.
Und deswegen sind die beiden Verfahren �quivalent.
