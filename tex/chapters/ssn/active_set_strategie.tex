In diesem Unterkapitel wird
die Aktive-Mengen-Strategie betrachtet.
Dieses Verfahren ist interessant,
weil man es in der Praxis als
die Implementierung des halbglatten Newton-Verfahrens nehmen kann.
Es hat eine gemeinsame Idee
mit dem Aktiven-Mengen-Verfahren in Abschnitt~\ref{sec:aktive_mengen_verfahren}.

Als Literatur �ber dieses Verfahren sei
auf \cite[Kapitel~7, S.~189~ff.]{ito} verwiesen.
Es wurde in \cite[Abschnitt~8.4, S.~240~ff.]{ito}
gezeigt, dass es als halbglattes
Newton-Verfahren interpretiert werden kann.

Es ist hier das
Problem~\eqref{prob:opt_prob_mit_lin_ungl_nebenbed}
\begin{align}
  \min_{x \in \R^n}\ & f(x)
    \tag{PLU}\\
  \nb & Ax = b \notag \\
      & G x \leq r \notag
\end{align}
zu betrachten.

Eine der Optimalit�tsbedingungen dieses Problems ist
nach Satz~\ref{satz:karush_kuhn_tucker}
\begin{equation}
\label{eq:erste_opt_bed_fuer_opt_prob_mit_lin_ungl_nebenbed}
  \nabla f(x) + A^T \lambda + G^T \mu = 0.
\end{equation}
Dies ist ein nichtlineares Gleichungssystem, wenn $f$ nichtlinear ist.

Sei $x^k$ ein Punkt in der N�he der L�sung des Problems.
Die Approximation der Funktion $f$ in der Umgebung von $x^k$ lautet
\begin{equation}
  f(x) \approx f(x^k) + \nabla f(x^k)^T (x-x^k)
                 + \frac{1}{2} (x-x^k)^T f''(x^k) (x-x^k).
\end{equation}
Die Approximation des Gradienten von $f$ ist folglich
\begin{equation}
  \nabla f(x) \approx \nabla f(x^k) + f''(x^k) x - f''(x^k) x^k.
\end{equation}

Ersetzt man $\nabla f(x)$ in der
Bedingung~\eqref{eq:erste_opt_bed_fuer_opt_prob_mit_lin_ungl_nebenbed}
mit seiner Approximation, so l�sst sie sich schreiben als 
\begin{align}
  && \nabla f(x^k) + f''(x^k) x - f''(x^k) x^k + A^T \lambda + G^T \mu
    &= 0 \\
  &\Leftrightarrow& f''(x^k) x + A^T \lambda + G^T \mu
    &= f''(x^k) x^k - \nabla f(x^k).
\end{align}
Dieses lineare Gleichungssystem wird in der Aktive-Mengen-Strategie verwendet.

\begin{algorithm}
\label{algo:aktive_mengen_strategie}
\emph{(Aktive-Mengen-Strategie
f�r~\eqref{prob:opt_prob_mit_lin_ungl_nebenbed})}
\begin{enumerate}
  \item W�hle $x^0$ und setze $\lambda^0 := 0,\: \mu^0 := 0,\: k := 0$. 
  \item Falls
        \begin{align*}
          \nabla f(x^k) + A^T \lambda^k + G^T \mu^k &= 0 \\
          Ax^k &= b \\
          \min\{\mu^k, r - G x^k\} &= 0
        \end{align*}
        $\Rightarrow$ STOP.
        \label{list:stop_criteria_in_active_set_strategy}
  \item Bestimme
        \begin{align}
        \A_k & := \{i \in \{1,\ldots,p\}\ |\ 
          r_i - \langle g_i, x^k \rangle < \mu_i^k \}, \\
        \I_k & := \{i \in \{1,\ldots,p\}\ |\ 
          r_i - \langle g_i, x^k \rangle \geq \mu_i^k \}.
        \end{align}
  \item L�se das Problem
        \begin{align}
          f''(x^k) x + A^T \lambda + G^T \mu
            &= f''(x^k) x^k - \nabla f(x^k) \label{eq:eq1_in_ass} \\
          A x &= b \label{eq:eq2_in_ass} \\
          \langle g_i, x \rangle &= r_i \quad \text{ f�r } i \in \A_k
            \label{eq:eq3_in_ass} \\
          \mu_i & = 0   \quad \text{ f�r } i \in \I_k.
            \label{eq:eq4_in_ass}
        \end{align}
        $\Rightarrow$ Erhalte $x, \lambda$ und $\mu$.
  \item Setze $x^{k+1} := x,\ \mu^{k+1} := \mu$ und $k := k+1$
        $\Rightarrow$ Gehe zu
        Schritt~\ref{list:stop_criteria_in_active_set_strategy}.
\end{enumerate}
\end{algorithm}

Das Verfahren~\ref{algo:aktive_mengen_strategie} kann
als halbglattes Newton-Verfahren
(Verfahren~\ref{algo:halbglattes_newton_fuer_restr_opt_prob})
interpretiert werden.
Sei $k$ irgendein Iterationsschritt in beiden Verfahren.
Bei dem halbglatten Newton-Verfahren ist dann
in Schritt~\ref{list:gleichungssystem_in_halbgl_newton_fuer_PLU}
folgendes Gleichungssystem zu l�sen:
\begin{equation*}
  \left(\begin{array}{ccc}
         f''(x)         &  A^T   &      G^T             \\
           A            &   0    &       0              \\
    - \chi_\A(1)\, g_1^T  &   0    & \chi_\I(1)\, e_1^T \\
        \vdots          & \vdots &    \vdots            \\
    - \chi_\A(p)\, g_p^T  &   0    & \chi_\I(p)\, e_p^T
  \end{array}\right)
  \left(\begin{array}{c} d_x \\ d_\lambda \\ d_\mu \end{array}\right)
  = -
  \left(\begin{array}{c}
    \nabla f(x^k) + A^T \lambda^k + G^T \mu^k \\
    Ax^k - b \\
    \chi_\I(1)\, \mu_1^k + \chi_\A(1)\,
      \bigl( r_1 - \langle g_1, x^k \rangle \bigr) \\
    \vdots \\
    \chi_\I(p)\, \mu_p^k + \chi_\A(p)\,
      \bigl( r_p - \langle g_p, x^k \rangle \bigr)
  \end{array}\right),
\end{equation*}
d.\,h.
\begin{align}
  f''(x^k) d_x + A^T d_\lambda + G^T d_\mu &
    = - \nabla f(x^k) - A \lambda^k - G^T \mu^k
    \label{eq:eq1_in_ssn} \\
  A d_x & = - A x^k + b \label{eq:eq2_in_ssn} \\
  - \chi_{\A}(i)\, g_i^T d_x + \chi_{\I}(i)\, e_i^T d_\mu
    & = -\chi_{\I}(i)\, \mu_i^k - \chi_{\A}(i)\,
    \bigl( r_i - \langle g_i, x^k \rangle \bigr)
    \quad i = 1,\ldots,p \label{eq:eq3_in_ssn}
\end{align}

Die Indexmengen $\A$ und $\I$ sind dabei genau wie
die in der Aktive-Mengen-Strategie.

Bei dem halbglatten Newton-Verfahren
werden in Schritt~\ref{list:update_in_halbgl_newton_fuer_PLU}
die Variablen $x^{k+1},\: \lambda^{k+1}$ und $\mu^{k+1}$ gesetzt.
Es gilt hiernach
\begin{equation}
  d_x = x^{k+1} - x^k, \quad
  d_\lambda = \lambda^{k+1} - \lambda^k, \quad \text{und} \quad
  d_\mu = \mu^{k+1} - \mu^k.
\end{equation}

Dann lautet Gleichung~\eqref{eq:eq1_in_ssn}
\begin{align}
  f''(x^k) (x^{k+1} - x^k) + A^T (\lambda^{k+1} - \lambda^k)
    + G^T (\mu^{k+1} - \mu^k) & = -\nabla f(x^k) - A^T \lambda^k - G^T \mu^k
    \notag \\
  f''(x^k) x^{k+1} + A^T \lambda^{k+1} + G^T \mu^{k+1} & =
    f''(x^k) x^k - \nabla f(x^k).
\end{align}
Das ist genau die Gleichung~\eqref{eq:eq1_in_ass}
in der Aktive-Mengen-Strategie
mit $x^{k+1},\, \lambda^{k+1},\, \mu^{k+1}$
anstatt $x,\, \lambda,\, \mu$.

F�r die Gleichung~\eqref{eq:eq2_in_ssn} gilt
\begin{align}
  A (x^{k+1} - x^k) & = - A x^k + b \\
  A x^{k+1} & = b.
\end{align}
Sie entspricht also der Gleichung~\eqref{eq:eq2_in_ass}
in Verfahren~\ref{algo:aktive_mengen_strategie}.

Sei $i \in \A_k$.
Die Gleichung~\eqref{eq:eq3_in_ssn}
l�sst sich somit schreiben als
\begin{align}
  - g_i^T d_x & = - \bigl( r_i - \langle g_i, x^k \rangle \bigr) \\
  - g_i^T (x^{k+1} - x^k) & = - r_i + \langle g_i, x^k \rangle \\
  r_i & = \langle g_i, x^{k+1} \rangle.
\end{align}
Sie findet man auch in der Gleichung~\eqref{eq:eq3_in_ass}
in der Aktive-Mengen-Strategie.

Sei nun $i \in \I_k$.
Dann sieht die Gleichung~\eqref{eq:eq3_in_ssn} wie folgt aus.
\begin{align}
  e_i^T d_\mu & = - \mu_i^k \\
  e_i^T (\mu^{k+1} - \mu^k) & = - \mu_i^k \\
  \mu^{k+1}_i - \mu^k_i & = - \mu_i^k \\
  \mu^{k+1}_i & = 0.
\end{align}
Das ist die Gleichung~\eqref{eq:eq4_in_ass}
in der Aktive-Mengen-Strategie mit $\mu^{k+1}_i$ anstatt $\mu_i$.

D.\,h., die beiden Verfahren berechnen eigentlich
denselben n�chsten Iterationspunkt $x^{k+1}$.
Deswegen kann die Aktive-Mengen-Strategie als
das halbglatte Newton-Verfahren interpretiert werden.