Wir betrachten in diesem Unterkapitel
die Aktive-Menge-Strategie.
Dieses Verfahren ist f�r uns interessant,
weil sie als das halbglatte Newton-Verfahren interpretiert werden kann.

Wir betrachten wieder das Problem
\begin{align}
  \min_{x \in \R^n}\ & f(x)
    \tag{PLU} \label{} \\
  \nb & G x \leq r \notag
\end{align}

Eine der Optimalit�tsbedingungen dieses Problems ist:
\begin{equation}
  \nabla f(x) + G^T \mu = 0.
\end{equation}
Das ist ein nichtlineares Gleichungssystem, wenn $f$ nichtlinear ist.

Wir versuchen nun, eine linearisierte Version zu bekommen.

Sei $x^k$ ein zul�ssiger Punkt des Problems.
Approximieren wir die Funktion $f$ in der Umgebung von $x^k$:
\begin{equation}
  f(x) \approx f(x^k) + \nabla f(x^k)^T (x-x^k)
                 + \frac{1}{2} (x-x^k)^T f''(x^k) (x-x^k).
\end{equation}
Der Gradient von $f$ ist dann
\begin{equation}
  \nabla f(x) \approx \nabla f(x^k) + f''(x^k) x - f''(x^k) x^k.
\end{equation}

Ersetzen wir nun $\nabla f(x)$ in der Optimalit�tsbedingung mit seiner
Approximation, dann bekommen wir die Gleichung
\begin{align}
  \nabla f(x^k) + f''(x^k) x - f''(x^k) x^k + G^T \mu
    &= 0 \\
  f''(x^k) x + G^T \mu
    &= f''(x^k) x^k - \nabla f(x^k)
\end{align}

Das ist nun ein lineares Gleichungssystem.
Dieses werden wir in unserer Aktive-Mengen-Strategie verwenden.

\begin{algorithm}
\emph{(Aktive-Menge-Strategie)}
\begin{enumerate}
  \item W�hle $x^0$ und setze $k := 0$. 
  \item Bestimme
        \begin{align}
        \A^k & := \{i \in \{1,\ldots,p\}\ |\ r_i - g_i^T x^k < \mu_i \}, \\
        \I^k & := \{i \in \{1,\ldots,p\}\ |\ r_i - g_i^T x^k \geq \mu_i \}
        \end{align}
  \item L�se das Problem
        \begin{align}
          f''(x^k) x + G^T \mu^{k+1}
                      &= f''(x^k) x^k - \nabla f(x^k) \\
          g_i^T x &= r_i \quad \text{ f�r } i \in \A^k \\
          \mu_i & = 0   \quad \text{ f�r } i \in \I^k
        \end{align}
  \item Falls
        \begin{align}
          \nabla f(x^{k+1}) + \mu & = 0\\
          \min(\mu^{k+1}, r - G x^{k+1}) & = 0
        \end{align}
        $\Rightarrow$ STOP
  \item $x^k := x^{k+1}, k := k+1$ $\Rightarrow$ Gehe zu Schritt 2. %TODO ref
\end{enumerate}
\end{algorithm}

