In diesem Unterkapitel wird
die Aktive-Mengen-Strategie betrachtet.
Dieses Verfahren ist interessant,
weil man es in der Praxis als
die Implementierung des halbglatten Newton-Verfahrens nimmt.
Es hat eine gemeinsame Idee
mit dem Aktiven-Mengen-Verfahren in Abschnitt~\ref{sec:aktive_mengen_verfahren}.

Es ist hier das
Problem~\eqref{prob:opt_prob_mit_lin_ungl_nebenbed}
\begin{align}
  \min_{x \in \R^n}\ & f(x)
    \tag{PLU}\\
  \nb & Ax = b \notag \\
      & G x \leq r \notag
\end{align}
zu betrachten.

Eine der Optimalit�tsbedingungen dieses Problems ist
\begin{equation}
\label{eq:erste_opt_bed_fuer_opt_prob_mit_lin_ungl_nebenbed}
  \nabla f(x) + A^T \lambda + G^T \mu = 0.
\end{equation}
Das ist ein nichtlineares Gleichungssystem, wenn $f$ nichtlinear ist.
Es wird nun hier versucht, eine linearisierte Version zu erhalten.

Sei $x^k$ ein Punkt in der N�he der L�sung des Problems.
Die Approximation der Funktion $f$ in der Umgebung von $x^k$ lautet
\begin{equation}
  f(x) \approx f(x^k) + \nabla f(x^k)^T (x-x^k)
                 + \frac{1}{2} (x-x^k)^T f''(x^k) (x-x^k).
\end{equation}
Die Approximation des Gradienten von $f$ ist folglich
\begin{equation}
  \nabla f(x) \approx \nabla f(x^k) + f''(x^k) x - f''(x^k) x^k.
\end{equation}

Ersetzt man $\nabla f(x)$ in der
Bedingung~\eqref{eq:erste_opt_bed_fuer_opt_prob_mit_lin_ungl_nebenbed}
mit seiner Approximation, dann l�sst sich sie schreiben als
das 
\begin{align}
  \nabla f(x^k) + f''(x^k) x - f''(x^k) x^k + A^T \lambda + G^T \mu
    &= 0 \\
  f''(x^k) x + A^T \lambda + G^T \mu
    &= f''(x^k) x^k - \nabla f(x^k).
\end{align}
Dieses lineare Gleichungssystem wird in der Aktive-Mengen-Strategie verwendet.

\begin{algorithm}
\label{algo:aktive_mengen_strategie}
\emph{(Aktive-Mengen-Strategie
f�r~\eqref{prob:opt_prob_mit_lin_ungl_nebenbed})}
\begin{enumerate}
  \item W�hle $x^0$ und setze $\mu^0 := 0,\: k := 0$. 
  \item Bestimme
        \begin{align}
        \A_k & := \{i \in \{1,\ldots,p\}\ |\ 
          r_i - \langle g_i, x^k \rangle < \mu_i^k \}, \\
        \I_k & := \{i \in \{1,\ldots,p\}\ |\ 
          r_i - \langle g_i, x^k \rangle \geq \mu_i^k \}.
        \end{align}
        \label{list:bestimmung_A_und_I_in_active_set_strategy}
  \item L�se das Problem
        \begin{align}
          f''(x^k) x + A^T \lambda + G^T \mu
            &= f''(x^k) x^k - \nabla f(x^k) \label{eq:eq1_in_ass} \\
          A x &= b \\
          \langle g_i, x \rangle &= r_i \quad \text{ f�r } i \in \A_k
            \label{eq:eq2_in_ass} \\
          \mu_i & = 0   \quad \text{ f�r } i \in \I_k.
            \label{eq:eq3_in_ass}
        \end{align}
        $\Rightarrow$ Erhalte $x, \lambda$ und $\mu$.
  \item Falls
        \begin{align}
          \nabla f(x) + A^T \lambda + G^T \mu & = 0\\
          \min\{\mu, r - G x\} & = 0
        \end{align}
        $\Rightarrow$ STOP.
  \item Setze $x^{k+1} := x,\ \mu^{k+1} := \mu$ und $k := k+1$
        $\Rightarrow$ Gehe zu
        Schritt~\ref{list:bestimmung_A_und_I_in_active_set_strategy}.
\end{enumerate}
\end{algorithm}

Es kann gezeigt werden, dass das Verfahren~\ref{algo:aktive_mengen_strategie}
als das halbglatte Newton-Verfahren
(Verfahren~\ref{algo:halbglattes_newton_fuer_restr_opt_prob}) interpretiert
werden kann.

Sei $k$ irgendeiner Iterationsschritt in den beiden Verfahren.
Bei dem halbglatten Newton-Verfahren ist dann
folgendes Gleichungssystem zu l�sen:
\begin{equation}
  \left(\begin{array}{ccc}
         f''(x)         &  A^T   &      G^T             \\
           A            &   0    &       0              \\
    - \chi_\A(1)\, g_1^T  &   0    & \chi_\I(1)\, e_1^T \\
        \vdots          & \vdots &    \vdots            \\
    - \chi_\A(p)\, g_p^T  &   0    & \chi_\I(p)\, e_p^T
  \end{array}\right)
  \left(\begin{array}{c} d_x \\ d_\lambda \\ d_\mu \end{array}\right)
  = -
  \left(\begin{array}{c}
    \nabla f(x^k) + A^T \lambda^k + G^T \mu^k \\
    \chi_{\I_k}(1) \mu_1 + \chi_{\A_k}(1) (r_1 - g_1^T x^k) \\
      \vdots \\
    \chi_{\I_k}(p) \mu_p + \chi_{\A_k}(p) (r_p - g_p^T x^k)
  \end{array}\right)
\end{equation}

D.\,h.
\begin{align}
  f''(x^k) d_x + G^T d_\mu & = -\nabla f(x^k) - G^T \mu^k \\
  - \chi_{\A_k}(j) g_j^T d_x + \chi_{\I_k}(j) e_j^T d_\mu
    & = -\chi_{\I_k}(j) \mu_j - \chi_{\A_k}(j) (r_j - g_j^T x^k)
    \quad \text{f�r } j = 1,\ldots,p \label{eq:eq2in_ssn}
\end{align}

Die Indexmengen $\A_k$ und $\I_k$ sind dabei genau wie
in der Aktive-Mengen-Strategie.

Bei dem halbglatten Newton-Verfahren setzen wir dann
$x^{k+1} := x^k + d_x$ und $\mu^{k+1} := \mu^k + d_\mu$.
Es gilt also $d_x = x^{k+1} - x^k$ und $d_\mu = \mu^{k+1} - \mu^k$.
D.\,h.
\begin{align}
  f''(x^k) (x^{k+1} - x^k) + G^T (\mu^{k+1} - \mu^k)
    & = -\nabla f(x^k) - G^T \mu^k \\
  \Rightarrow \qquad f''(x^k) x^{k+1} + G^T \mu^{k+1} & =
    f''(x^k) x^k - \nabla f(x^k)
\end{align}
Das ist genau die Gleichung~\eqref{eq:eq1_in_ass}
in der Aktive-Mengen-Strategie.

Sei $j \in \A_k$.
Die Gleichung~\eqref{eq:eq2in_ssn} ist dann gleich
\begin{align}
  - g_j^T d_x & = - (r_j - g_j^T x^k) \\
  \Rightarrow - g_j^T (x^{k+1} - x^k) & = - (r_j - g_j^T x^k) \\
  \Rightarrow   g_j^T x^{k+1} & = r_j
\end{align}
Das ist genau die Gleichung~\eqref{eq:eq2_in_ass}
in der Aktive-Mengen-Strategie.

Sei nun $j \in \I_k$.
Die Gleichung~\eqref{eq:eq2in_ssn} ist dann gleich
\begin{align}
  e_j^T d_\mu & = - \mu_j \\
  \Rightarrow e_j^T (\mu^{k+1} - \mu^k) & = - \mu_j \\
  \Rightarrow \mu^{k+1}_j - \mu^k_j & = - \mu_j \\
  \Rightarrow \mu^{k+1}_j & = 0
\end{align}
Das ist genau die Gleichung~\eqref{eq:eq3_in_ass}
in der Aktive-Mengen-Strategie.

Somit sind die Iterationsschritte in den beiden Verfahren gleich.
Und deswegen sind die beiden Verfahren �quivalent.
