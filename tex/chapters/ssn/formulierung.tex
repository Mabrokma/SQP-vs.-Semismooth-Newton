Es wird nun die Formulierung des halbglatten Newton-Verfahrens vorgestellt
um nichtlineare Optimierungsprobleme mit linearen Restriktionen zu l�sen.

F�r eine L�sung $\xopt$
des Problems~\ref{prob:opt_prob_mit_lin_ungl_nebenbed}
\begin{align}
  \min_{x \in \R^n}\ & f(x)
    \tag{PLU}\\
  \nb & Ax = b \notag \\
      & G x \leq r \notag
\end{align}
gelten nach Satz~\ref{satz:karush_kuhn_tucker}
die Optimalit�tsbedingungen
\begin{gather}
  \nabla f(\xopt) + A^T \lambda + G^T \mu = 0 \\
  A \xopt = b \\
  \mu_j \geq 0, \quad \langle g_j, \xopt \rangle \leq r_j, \quad
    \mu_j (r_j - \langle g_j, \xopt \rangle) = 0 \quad
    \text{f�r } j = 1,\ldots,p
    \label{eq:optimalitaetsbed_in_prob_mit_lin_ungl_nebenbed}
\end{gather}
mit Vektoren $\lambda \in \R^m$ und $\mu \in \R^p$.

Gegeben seien $x_1, x_2 \in \R$.
Es gilt
\begin{equation}
  \begin{array}{c}
    x_1, x_2 \geq 0 \\
    x_1 x_2 = 0
  \end{array}
  \qquad \Leftrightarrow \qquad
  \min\{x_1,x_2\} = 0.
\end{equation}

Die Bedingungen~\eqref{eq:optimalitaetsbed_in_prob_mit_lin_ungl_nebenbed}
sind somit �quivalent zu
\begin{equation}
  \min\left\{ \mu_j, r_j - \langle g_j, \xopt \rangle \right\} = 0 \quad
    \text{f�r } j = 1,\ldots,p,
\end{equation}
oder in Matrixschreibweise:
\begin{equation}
  \min\left\{ \mu, r - G \xopt \right\} = 0,
\end{equation}
wobei der Operator $\min$ hier dann elementenweise arbeitet.

Deswegen k�nnen die Optimalit�tsbedingungen wie folgt
geschrieben werden.
\begin{align}
  \nabla f(\xopt) + A^T \lambda + G^T \mu & = 0 \\
  A \xopt & = b \\
  \min\left\{ \mu, r - G \xopt \right\} & = 0
\end{align}.

Sei nun die Funktion
\begin{equation}
  F(x,\lambda,\mu) :=
  \left(\begin{array}{c}
    \nabla f(x) + A^T \lambda + G^T \mu \\
    Ax - b \\
    \min\left\{ \mu, r - G \xopt \right\}
  \end{array}\right).
\end{equation}

Seien die Indexmengen
\begin{align}
  \A & := \{\ j \in \{1,\ldots,p\}\ |\ r_j - \langle g_j, x \rangle < \mu_j\ \}, \\
  \I & := \{\ j \in \{1,\ldots,p\}\ |r_j - \langle g_j, x \rangle \geq \mu_j\ \}
    = \{1,\ldots,p\} \backslash \A.
\end{align}
Mit
\begin{equation}
  \chi_K(k) :=
    \begin{cases}
      1 & \text{f�r } k \in K, \\
      0 & \text{sonst}
    \end{cases}
\end{equation}
gilt dann
\begin{equation}
  \min\left\{ \mu, r - G \xopt \right\} =
  \left(\begin{array}{c}
    \chi_\I(1) \mu_1 + \chi_\A(1) (r_1 - \langle g_1, x \rangle) \\
      \vdots \\
    \chi_\I(p) \mu_p + \chi_\A(p) (r_p - \langle g_p, x \rangle)
  \end{array}\right).
\end{equation}

Die Ableitung von $F$ lautet somit
\begin{equation}
  F'(x,\lambda,\mu) =
  \left(\begin{array}{ccc}
         f''(x)         &  A^T   &      G^T         \\
           A            &   0    &       0          \\
    - \chi_\A(1) g_1^T  &   0    & \chi_\I(1) e_1^T \\
        \vdots          & \vdots &    \vdots        \\
    - \chi_\A(p) g_p^T  &   0    & \chi_\I(p) e_p^T
  \end{array}\right)
\end{equation}

\begin{algorithm}
\emph{(Das halbglatte Newton-Verfahren)}
\begin{enumerate}
  \item W�hle $x^0, \lambda^0, \mu^0$ und setze $k := 0$.
  \item Berechne die L�sung \label{list:equation_ssn}
        $d = \left(\begin{array}{c} d_x \\ d_\lambda \\ d_\mu \end{array}\right)$
        des linearen Gleichungssystems
        \begin{equation}
          F'(x^k,\lambda^k,\mu^k) d = - F(x^k,\lambda^k,\mu^k).
        \end{equation}
        Setze $d^k_x := d_x, d^k_\lambda := d_\lambda$ und $d^k_\mu := d_\mu$.
  \item Ist $d^k_x = 0$
        $\Rightarrow$ STOP.
  \item Setze $x^{k+1} := x^k + d^k_x$,
        $\lambda^{k+1} := \lambda^k + d^k_\lambda$,
        $\mu^{k+1} := \mu^k + d^k_\mu$
        und $k := k+1$
        $\Rightarrow$ Gehe zu Schritt~\ref{list:equation_ssn}.
\end{enumerate}
\end{algorithm}