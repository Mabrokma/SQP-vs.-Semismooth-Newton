
Wir werden jetzt das halbglatte Newton-Verfahren formulieren,
um nichtlineare Optimierungsprobleme mit linearen Restriktionen
zu l�sen.

Gegeben seien $x_1, x_2 \in \R$.
Es gilt
\begin{equation}
  \begin{array}{c}
    x_1, x_2 \geq 0 \\
    x_1 x_2 = 0
  \end{array}
  \qquad \Leftrightarrow \qquad
  \min(x_1,x_2) = 0
\end{equation}

Wenn wir also die Bedingungen
\begin{equation}
  \mu \geq 0, \quad Gx \leq r \quad \text{und} \quad \mu^T(Gx-r) = 0
\end{equation}
haben, k�nnen wir sie als
\begin{equation}
  \min( \mu, r-Gx ) = 0
\end{equation}
schreiben. Hier arbeitet der Operator $\min$ elementenweise.

Wir betrachten nun das Problem
\begin{align}
  \min_{x \in \R^n}\ & f(x)
    \tag{PLU} \label{prob:opt_prob} \\
  \nb & G x \leq r \notag
\end{align}

Die Optimalit�tsbedingungen f�r dieses Problem ist
\begin{align}
  \nabla f(x) + G^T \mu & = 0 \\
  \mu \geq 0, \quad Gx \leq r, \quad \mu^T(Gx-r) & = 0
\end{align}

Diese k�nnen wir nach unserer vorherigen �berlegung schreiben als
\begin{align}
  \nabla f(x) + G^T \mu & = 0 \\
  \min( \mu, r-Gx ) & = 0
\end{align}

Wir definieren nun die Funktion
\begin{equation}
  F(x,\mu) =
  \left(\begin{array}{c}
    \nabla f(x) + G^T \mu \\
    \min( \mu, r-Gx )
  \end{array}\right)
\end{equation}

Wir m�ssen nur noch die Ableitung von $F$ bestimmen.
Dann k�nnen wir das Verfahren formulieren.

Wir definieren:
\begin{align}
  \A & := \{\ j \in \{1,\ldots,p\}\ |\ r_j - g_j^T x < \mu_j\ \}, \\
  \I & := \{\ j \in \{1,\ldots,p\}\ |\ r_j - g_j^T x \geq \mu_j\ \} =
    \{1,\ldots,p\} \backslash \A
\end{align}
\begin{equation}
  \chi_M(m) :=
    \begin{cases}
      1 & \text{f�r } m \in M, \\
      0 & \text{sonst}.
    \end{cases}
\end{equation}

D. h. es gilt
\begin{equation}
  \min( \mu, r-Gx ) =
  \left(\begin{array}{c}
    \chi_\I(1) \mu_1 + \chi_\A(1) (r_1 - g_1^T x) \\
      \vdots \\
    \chi_\I(p) \mu_p + \chi_\A(p) (r_p - g_p^T x)
  \end{array}\right)
\end{equation}

Die Ableitung ist dann in der Form:
\begin{equation}
  F'(x,\mu) =
  \left(\begin{array}{cc}
         f''(x)         &       G^T \\
    - \chi_\A(1) g_1^T  &  \chi_\I(1) e_1^T \\
        \vdots          &      \vdots \\
    - \chi_\A(p) g_p^T  &  \chi_\I(p) e_p^T
  \end{array}\right)
\end{equation}

\begin{algorithm}
\emph{(Das halbglatte Newton-Verfahren)}
\begin{enumerate}
  \item W�hle $x^0$, $\mu^0$
        und ein Abbruchkriterium $\epsilon > 0$.
        Setze $k := 0$.
  \item Berechne die L�sung \label{list:equation_ssn}
        $d = \left(\begin{array}{c} d_x \\ \mu \end{array}\right)$
        des linearen Gleichungssystems
        \begin{equation}
          F'(x^k,\mu^k) d = - \nabla F(x^k,\mu^k).
        \end{equation}
        Setze $d^k := d_x$ und $\mu^{k+1} := \mu$.
  \item Ist $\|\nabla d^k\| < \epsilon$
        $\Rightarrow$ STOP.
  \item Setze $x^{k+1} := x^k + d^k$ und $k := k+1$ $\Rightarrow$
        Gehe zu Schritt~\ref{list:equation_ssn}.
\end{enumerate}
\end{algorithm}