
Wir werden jetzt das halbglatte Newton-Verfahren formulieren,
um nichtlineare Optimierungsprobleme mit linearen Restriktionen
zu l�sen.

Gegeben seien $x_1, x_2 \in \R$.
Es gilt
\begin{equation}
  \begin{array}{c}
    x_1, x_2 \geq 0 \\
    x_1 x_2 = 0
  \end{array}
  \qquad \Leftrightarrow \qquad
  \min(x_1,x_2) = 0
\end{equation}

Wenn wir also die Bedingungen
\begin{equation}
  \mu \geq 0, \quad Gx \leq r \quad \text{und} \quad \mu^T(Gx-r) = 0
\end{equation}
haben, k�nnen wir sie als
\begin{equation}
  \min( \mu, r-Gx ) = 0
\end{equation}
schreiben. Hier arbeitet der Operator $\min$ elementenweise.

Wir betrachten nun das Problem
\begin{align}
  \min_{x \in \R^n}\ & f(x)
    \tag{PLU} \label{prob:opt_prob} \\
  \nb & G x \leq r \notag
\end{align}

Die Optimalit�tsbedingungen f�r dieses Problem ist
\begin{align}
  \nabla f(x) + G^T \mu & = 0 \\
  \mu \geq 0, \quad Gx \leq r, \quad \mu^T(Gx-r) & = 0
\end{align}

Diese k�nnen wir nach unserer vorherigen �berlegung schreiben als
\begin{align}
  \nabla f(x) + G^T \mu & = 0 \\
  \min( \mu, r-Gx ) & = 0
\end{align}

Wir definieren nun die Funktion
\begin{equation}
  F(x,\mu) =
  \left(\begin{array}{c}
    \nabla f(x) + G^T \mu \\
    \min( \mu, r-Gx )
  \end{array}\right)
\end{equation}

Wir m�ssen nur noch die Ableitung von $F$ bestimmen.
Dann k�nnen wir das Verfahren formulieren.

Die Ableitung ist in der Form:
\begin{equation}
  F'(x,\mu) =
  \left(\begin{array}{c}
    f''(x) \quad G^T \mu \\
    W
  \end{array}\right)
\end{equation}

Die Matrix $W$ ist von dem Ergebnis von $\min( \mu, r-Gx )$ abh�ngig.

