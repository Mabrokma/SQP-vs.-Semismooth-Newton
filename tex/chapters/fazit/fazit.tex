\chapter{Fazit und Ausblick}

Es gibt mehrere Kriterien, um die Effizienz eines Verfahrens zu bewerten.
Die Kriterien sind z.\,B. die Laufzeit des Programms und
die Anzahl der ben�tigten Iterationen.
Hier bei dem halbglatten Newton-Verfahren
und SQP-Verfahren ziehen wir daher folgende Schlussfolgerungen:
\begin{itemize}
  \item Falls die Laufzeit das wichtigste Kriterium sein soll, dann ist
  halbglattes Newton-Verfahren das beste Verfahren f�r Probleme
  mit wenigen Variablen und SQP-Verfahren das bessere Verfahren f�r
  Probleme mit vielen Variablen.
  \item Falls die Anzahl der Iterationen eine wichtigere Rolle spielt,
  dann ist SQP-Verfahren das beste Verfahren.
\end{itemize}
Wenn man die Implementierung des Verfahrens
ber�cksichtigt, dann ist halbglattes Newton-Verfahren
die bessere Wahl, da die Implementierung von SQP-Verfahren
aufwendig sein kann.

In dieser Arbeit wurden keine Testprobleme mit nichtlinearen Restriktionen
ber�cksichtigt. Es w�re interessant, zu sehen, wie die Ergebnisse
bei diesen Testproblemen aussehen.
Ein anderer interessanter Aspekt ist
die Eigenschaft der globalen Konvergenz beider Verfahren.
F�r jedes Testproblem wurde hier der Startpunkt in der N�he der L�sung
angegeben, damit die Eigenschaft der lokalen Konvergenz beider Verfahren
verwendet werden konnte.
Die beiden Verfahren k�nnen jedoch noch erweitert werden,
damit ein beliebiger Startpunkt benutzt werden kann.
Diese Varianten beider Verfahren mit globaler Konvergenz
sollten noch untersucht und verglichen werden.

Eine Idee von Ulbrich~\cite{ulbrich} ist,
dass man beide Verfahren kombiniert.
Man kann versuchen, das Teilproblem in SQP-Verfahren
mit halbglattem Newton-Verfahren
anstatt mit der Aktive-Mengen-Verfahren
zu l�sen.
Es sollte untersucht werden,
wie konkurrenzf�hig diese Kombination ist.
