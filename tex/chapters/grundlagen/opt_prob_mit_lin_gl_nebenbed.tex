\begin{definition}
\emph{(Optimierungsprobleme mit linearen Gleichungsnebenbedingungen)}
\begin{align}
  \min_{x \in \R^n}\ & f(x) \tag{PLG}\label{prob:opt_prob_mit_lin_gl_nebenbed}\\
              \nb & Ax = b \notag
\end{align}
$A$ sei eine $(m \times n)$-Matrix und $b$ sei ein Vektor mit $m$ Elementen.
\end{definition}

D.\,h., die Menge $\F$ sieht hier so aus:  $\F = \{ x \in \R^n\ |\ A x = b \}$.

\begin{theorem} \label{satz:notw_bed_fuer_prob_mit_lin_gl_nebenbed}
\emph{(Notwendige Bedingung, vgl. Satz 5.3.2 in \cite[S.~174]{alt})}\\
Sei $\xopt$ eine lokale L�sung des
Problems~\eqref{prob:opt_prob_mit_lin_gl_nebenbed}
und $f$ sei in $\xopt$ differenzierbar.
Dann gibt es ein $\lambda \in \R^m$ mit
\begin{equation}
  \nabla f(\xopt) + A^T \lambda = 0.
\end{equation}
Hat A einen vollen Rang, dann ist $\lambda$ eindeutig bestimmt.
\end{theorem}

Diese Bedingung hei�t die Multiplikatoren-Regel von Lagrange. Man bezeichnet
$\lambda$ als Lagrange-Multiplikator.

\begin{theorem}
\emph{(Hinreichende Bedingung, vgl. Satz 5.3.8 in \cite[S.~177]{alt})}\\
Sei $f$ in $\xopt$ zweimal stetig differenzierbar. Die notwendige Bedingung
in Satz~\ref{satz:notw_bed_fuer_prob_mit_lin_gl_nebenbed} sei erf�llt.
Es existiere eine Konstante $\alpha > 0$ mit
\begin{equation}
  d^T f''(x) d \geq \alpha \|d\|^2 \qquad \forall d \in \kr A.
\end{equation}
Dann ist $\xopt$ eine strikte L�sung des
Problems~\eqref{prob:opt_prob_mit_lin_gl_nebenbed}.
\end{theorem}

Eine wichtige Grundlage zum L�sen des Pr�blems~\eqref{prob:opt_prob_mit_lin_gl_nebenbed}
als ein unrestringiertes Problem
ist die Definition der Nullraum-Matrix
(vgl. \cite[S.~182]{alt}).

\begin{definition}
\emph{(Nullraum-Matrix)}\\
Eine $(n \times l)$-Matrix $Z$ hei�t Nullraum-Matrix von~$A$, wenn f�r $d \in
\R^n$ gilt
\begin{equation}
d \in \kr A \quad \Leftrightarrow \quad d = Z z \text{ f�r ein } z \in \R^l.
\end{equation}
D.\,h. $\im Z = \kr A$.
\end{definition}

Sei $w$ eine L�sung von der Gleichung $Ax=b$.
Man kann nun f�r die Menge $\F$ des Problems~\eqref{prob:opt_prob_mit_lin_gl_nebenbed}
so schreiben:
\begin{equation}
  \F = w + \kr A = w + \im Z = w + \{ Z z | z \in \R^l \}.
\end{equation}

D.\,h., Jedes Element $x \in \F$ ist mit $w + Zz$, $z \in \R^l$, zu ersetzen.
Das Problem~\eqref{prob:opt_prob_mit_lin_gl_nebenbed} ist dann �quivalent zu
\begin{equation}
  \min_{z \in \R^l} F(z) := f(w + Zz).
\end{equation}

Dieses Problem hat keine Nebenbedingung mehr, also unrestringiert.
Man kann also Verfahren f�r unrestringierte Probleme anwenden.
Wir werden nachher im Unterkapitel~\ref{sec:nullraum_verfahren} sehen, wie
man dieses Problem effektiv l�sen kann, wenn die Zeilfunktion quadratisch ist.