\begin{definition}
\emph{(Optimierungsprobleme mit linearen Ungleichungsnebenbedingungen)}
\begin{align}
  \min_{x \in \R^n}\ & f(x)
    \tag{PLU} \label{prob:opt_prob_mit_lin_ungl_nebenbed} \\
                 \nb & Ax = b \notag \\
                     & Gx \leq r \notag
\end{align}
$A$ sei eine $(m \times n)$-Matrix mit $m \leq n$ und $b \in \R^m$.
$G$ sei eine $(p \times n)$-Matrix und $r \in \R^p$.
\end{definition}

D.\,h., die Menge $\F$ sieht hier so aus:
$\F = \{ x \in \R^n\ |\ A x = b,\ G x \leq r \}$.

Seien $a_k \in \R^n$, $k = 1,\ldots,m$, bzw. $g_j \in \R^n$, $j = 1,\ldots,p$,
Vektoren in der Matrix $A$ bzw. $G$, so dass
\begin{equation}
  A =
  \left(
    \begin{array}{c}
      a_1^T \\
      \vdots \\
      a_m^T
    \end{array}
  \right)
\quad \text{und} \quad
  G =
  \left(
    \begin{array}{c}
      g_1^T \\
      \vdots \\
      g_p^T
    \end{array}
  \right).
\end{equation}

Seien $b_k \in \R$, $k = 1,\ldots,m$, bzw. $r_j \in \R$, $j = 1,\ldots,p$,
die Elemente von $b$ bzw. $r$.
Dann k�nnen wir das Problem~\eqref{prob:opt_prob_mit_lin_ungl_nebenbed}
ausf�hrlicher schreiben.
\begin{align*}
  \min_{x \in \R^n}\ & f(x) \\
     \nb & \langle a_k, x \rangle = b_k \text{ f�r } k = 1,\ldots,m \\
         & \langle g_j, x \rangle \leq r_j \text{ f�r } j = 1,\ldots,p
\end{align*}

Die Notwendige Bedingung f�r eine lokale L�sung
des Problems~\eqref{prob:opt_prob_mit_lin_ungl_nebenbed}
ergibt sich aus dem folgenden Satz.

\begin{theorem} \label{satz:karush_kuhn_tucker}
\emph{(Karush-Kuhn-Tucker-Satz)}
Sei $\xopt$ lokale L�sung des Problems~\eqref{prob:opt_prob_mit_lin_ungl_nebenbed} und
$f$ sei in~$\xopt$ differenzierbar.
Dann existieren die Vektoren $\lambda \in \R^m$ und $\mu \in \R^p$
zu~$\xopt$ mit
\begin{align}
   \nabla f(\xopt) + A^T \lambda + G^T \mu & = 0 \\
  \mu_j (\langle g_j,\xopt \rangle - r_j ) & = 0 \quad \forall j = 1,\ldots,p \\
                                       \mu & \geq 0
\end{align}
$\lambda$ und $\mu$ hei�en Lagrange-Multiplikatoren zu $\xopt$.
\end{theorem}

Sei $x \in \F$. Wir bezeichnen mit
\begin{equation}
  J(x) := \{ 1 \leq j \leq p \ | \ \langle g_j,x \rangle = r_j \}
\end{equation}
die Indexmenge der in $x$ aktiven Ungleichungsrestriktionen.

\begin{theorem}
\emph{(Hinreichende Optimalit�tsbedingung)}
Sei $f$ in $\xopt \in \F$ zweimal stetig differenzierbar. Die notwendige
Bedingung von Satz~\ref{satz:karush_kuhn_tucker} sei erf�llt
und es gelte mit $\alpha > 0$
\begin{equation}
d^T f''(\xopt) d \geq \alpha \|d\|^2 \quad
  \forall d \in \R^n :
  \begin{cases}
    Ad = 0, & \\
    \langle g_j,d \rangle \leq 0
      & \text{f�r } j \in J(\xopt) \text{ mit } \mu_j = 0, \\
    \langle g_j,d \rangle = 0
      & \text{f�r } j \in J(\xopt) \text{ mit } \mu_j > 0.
  \end{cases}
\end{equation}
Dann ist $\xopt$ eine strikte lokale L�sung des Problems~\eqref{prob:opt_prob_mit_lin_ungl_nebenbed}
\end{theorem}

Spezialfall des Problems~\eqref{prob:opt_prob_mit_lin_ungl_nebenbed} ist das Optimierungsproblem mit
unteren und oberen Schranken f�r die Variablen.

\begin{definition}
\emph{(Optimierungsprobleme mit Variablenbeschr�nkungen)}
\begin{align}
  \min_{x \in \R^n}\ & f(x) \tag{PVB} \label{prob:opt_prob_mit_var_beschr} \\
                 \nb & a \leq x \leq b \notag
\end{align}
$a, b \in \R^n$ mit $a \leq b$.
\end{definition}

Die Nebenbedingung ist
�quivalent zu
$G x \leq r$
mit
$G := \left( \begin{array}{c} -I \\ I \end{array} \right)$ und
$r := \left( \begin{array}{c} -a \\ b \end{array} \right)$.

% TODO: sufficient condition?
