\begin{definition}
\emph{(Optimierungsproblem mit linearen Ungleichungsnebenbedingungen)}
\begin{align}
  \min_{x \in \R^n}\ & f(x)
    \tag{PLN} \label{prob:opt_prob_mit_lin_ungl_nebenbed} \\
                 \nb & Ax = b \notag \\
                     & Gx \leq r \notag
\end{align}
$A$ sei eine $(m \times n)$-Matrix mit $m \leq n$ und $b \in \R^m$.
$G$ sei eine $(p \times n)$-Matrix und $r \in \R^p$.
\end{definition}

D.\,h., die Menge $\F$ sieht hier so aus:
$\F = \{ x \in \R^n\ |\ A x = b,\ G x \leq r \}$.

Seien $g_j \in \R^n, j = 1,\ldots,p$, die Vektoren in der
Matrix $G$, so dass
\begin{equation}
  G =
  \left(
    \begin{array}{c}
      g_1^T \\
      \vdots \\
      g_p^T
    \end{array}
  \right).
\end{equation}
% TODO das Problem ausf�hrlicher schreiben

\begin{theorem} \label{satz:kkt}
\emph{(Karush-Kuhn-Tucker-Satz)}
Sei $\xopt$ lokale L�sung des Problems~\eqref{prob:opt_prob_mit_lin_ungl_nebenbed} und
$f$ sei in~$\xopt$ differenzierbar.
Dann existieren die Vektoren $\lambda \in \R^m$ und $\mu \in \R^p$
zu~$\xopt$ mit
\begin{align}
  \nabla f(\xopt) + A^T \lambda + G^T \mu & = 0 \\
  \mu \geq 0, \quad \mu_j (\langle g_j,\xopt \rangle - r_j ) & = 0,
    \quad j = 1,\ldots,p.
\end{align}
$\lambda$ und $\mu$ hei�en Lagrange-Multiplikatoren zu $\xopt$.
\end{theorem}

\begin{definition}
Sei $x \in \F$. Wir bezeichnen mit
\begin{equation}
  J(x) := \{ 1 \leq j \leq p \ | \ \langle g_j,x \rangle = r_j \}
\end{equation}
die Indexmenge der in $x$ aktiven Ungleichungsrestriktionen.
\end{definition}

\begin{theorem}
\emph{(Hinreichende Optimalit�tsbedingungen)}
Sei $f$ in $\xopt \in \F$ zweimal stetig differenzierbar. Die notwendige
Bedingung von Satz~\ref{satz:kkt} sei erf�llt und es gelte mit $\alpha > 0$
\begin{equation}
d^T f''(\xopt) d \geq \alpha \|d\|^2 \quad
  \forall d \in \R^n :
  \begin{cases}
    Ad = 0, & \\
    \langle g_j,d \rangle \leq 0
      & \text{f�r } j \in J(\xopt) \text{ mit } \mu_j = 0, \\
    \langle g_j,d \rangle = 0
      & \text{f�r } j \in J(\xopt) \text{ mit } \mu_j > 0.
  \end{cases}
\end{equation}
Dann ist $\xopt$ eine strikte lokale L�sung des Problems~\eqref{prob:opt_prob_mit_lin_ungl_nebenbed}
\end{theorem}

Spezialfall des Problems~\eqref{prob:opt_prob_mit_lin_ungl_nebenbed} ist das Optimierungsproblem mit
unteren und oberen Schranken f�r die Variablen.

\begin{definition}
\emph{(Optimierungsproblem mit Variablenbeschr�nkungen)}
\begin{align}
  \min_{x \in \R^n}\ & f(x) \tag{PVB} \label{prob:opt_prob_mit_var_beschr} \\
                 \nb & a \leq x \leq b \notag
\end{align}
$a, b \in \R^n$ mit $a \leq b$.
\end{definition}

Mit
$G := \left( \begin{array}{c} -I \\ I \end{array} \right)$ und
$r := \left( \begin{array}{c} -v \\ w \end{array} \right)$ hat $\F$ die Form
$\F := \{ x \in \R^n \ | \ G x \leq r \}$.

% TODO: sufficient condition?
