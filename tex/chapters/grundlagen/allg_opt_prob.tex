\begin{definition}
Allgemein ist die Aufgabenstellung der Optimierung wie folgt
definiert:
\begin{equation}
  \min_{x \in \F} f(x) \tag{P} \label{prob:allg_opt_prob}
\end{equation}
Die Funktion $f:D \subseteq \R^n \rightarrow \R$ ist die sogennante
Zielfunktion.
$D$ sei der Definitionsbereich von $f$.
$\F$ sei eine nichtleere Teilmenge von $D$, die man als L�sungsmenge bezeichnet.
Alle Elemente von $\F$ werden als zul�ssige Punkte bezeichnet.
$\F$ wird durch die sogennanten Nebenbedingungen definiert.
\end{definition}

Ein einfaches Beispiel ist das Problem
\[
  \min_{x \in \R}\ (x-1)^2.
\]

Falls $\F = D$ gilt,
dann bezeichnet man das Optimierungsproblem als unrestringiert.
Es besitzt also keine Nebenbedingungen.
Ansonsten hei�t es ein restringiertes Optimierungsproblem.

Man definiert in der Regel Optimierungsproblem als ein Minimierungsproblem,
weil ein Maximierungsproblem $\max g(x)$ zu dem Minimierungsproblem
$\min f(x) := -g(x)$ �quivalent ist.

\begin{definition}
\emph{(Globale und lokale L�sung)}
Ein Punkt $\xopt \in \F$ hei�t globale L�sung des Problems~\eqref{prob:allg_opt_prob} oder
globales Minimum, wenn
\begin{equation}
  f(\xopt) \leq f(x) \qquad \forall x \in \F
\end{equation}
gilt.
Ein Punkt $\xopt \in \F$ hei�t strikte globale L�sung des Problems~\eqref{prob:allg_opt_prob} oder
striktes globales Minimum, wenn
\begin{equation}
  f(\xopt) < f(x) \qquad \forall x \in \F\backslash\{\xopt\}
\end{equation}
gilt.
Ein Punkt $\xopt \in \F$ hei�t lokale L�sung des Problems~\eqref{prob:allg_opt_prob} oder
lokales Minimum, wenn f�r eine Umgebung $U(\xopt)$ von $\xopt$
\begin{equation}
  f(\xopt) \leq f(x) \qquad \forall x \in U(\xopt) \cap \F
\end{equation}
gilt.
Ein Punkt $\xopt \in \F$ hei�t strikte lokale L�sung des Problems~\eqref{prob:allg_opt_prob}
oder striktes lokales Minimum, wenn f�r eine Umgebung $U(\xopt)$ von
$\xopt$
\begin{equation}
  f(\xopt) < f(x) \qquad \forall x \in U(\xopt) \cap \F\backslash\{\xopt\}
\end{equation}
gilt.
Eine Umgebung $U(\xopt)$ von $\xopt$ ist eine offene Menge, die $\xopt$
beinhaltet.
\end{definition}

Wegen dieser Definitionen kommt der Begriff \emph{globale Optimierung}.
Bei der globalen Optimierung versucht man, eine globale L�sung zu finden.
Viele Verfahren versuchen nur lokale L�sungen zu bestimmen,
weil Globale L�sungen nicht so einfach zu bestimmen sind.

Die Aufgabenstellung bei der nichtlinearen Optimierung kann man
spezifischer wie folgt definieren:

\begin{definition}
\emph{(Nichtlineare Optimierungsprobleme)}
\begin{align}
  \min_{x \in D}\ & f(x) \tag{PN}\label{prob:opt_prob_mit_nichtlin_restr} \\
              \nb & g(x) \leq 0 \notag \\
                  & h(x) = 0 \notag
\end{align}
Die Zielfunktion ist wieder die Funktion $f:D \subseteq \R^n \rightarrow \R$.
Die Nebenbedingungen sind von den Funktionen
$g:D \subseteq \R^n \rightarrow \R^p$ und
$h:D \subseteq \R^n \rightarrow \R^m$
abh�ngig.
\end{definition}

D.\,h., die Menge $\F$ sieht hier so aus:
$\F = \{ x \in \R^n \,|\, g(x) \leq 0, \, h(x) = 0  \}$.

Man kann dieses Problem ausf�hrlicher schreiben, indem man die Funktionen
$g$ und $h$ in skalare Funktionen $g_1, \ldots, g_p : \R^n \rightarrow \R$
und $h_1, \ldots, h_m: \R^n \rightarrow \R$ zerlegen, so dass
\begin{equation*}
  g(x) =
    \left(
    \begin{array}{c}
      g_1(x) \\
      \vdots \\
      g_p(x)
    \end{array}
    \right)
\quad \text{und} \quad
  h(x) =
    \left(
    \begin{array}{c}
      h_1(x) \\
      \vdots \\
      h_m(x)
    \end{array}
    \right).
\end{equation*}

Man bekommt dann das Problem
\begin{align*}
  \min_{x \in D}\ & f(x) \\
              \nb & g_i(x) \leq 0 \text{ f�r } i = 1,\ldots,p \\
                  & h_j(x) = 0 \text{ f�r } j = 1,\ldots,m
\end{align*}

Man kann hierbei den Unterschied zwischen der linearen Optimierung und
der nichtlinearen Optimierung gut erkennen.
Bei der linearen Optimierung muss die Zielfunktion linear sein
(d.\,h., die Zielfunktion muss in der Form $f(x) = c^T x$, $c \in \R^n$, sein)
und die Nebenbedingungen sollen durch lineare Gleichungen bzw.
Ungleichungen definiert werden.
Bei der nichtlinearen Optimierung gibt es dagegen keine Einschr�nkung,
wie die Zielfunktion und die Nebenbedingungen aussehen sollen.

% TODO: Glatte bzw. nichtglatte Optimierung