\documentclass[%
	pdftex,%              PDFTex verwenden
	a4paper,%             A4 Papier
	oneside,%             Einseitig
	bibtotoc,%    		  Literaturverzeichnis einf�gen bibtotocnumbered: nummeriert
	liststotoc,%		  Verzeichnisse einbinden in toc
	idxtotoc,%            Index ins Verzeichnis einf�gen
	halfparskip,%         Europ�ischer Satz mit abstand zwischen Abs�tzen
	chapterprefix,%       Kapitel anschreiben als Kapitel
	headsepline,%         Linie nach Kopfzeile
	%footsepline,%        Linie vor Fusszeile
	%pointlessnumbers,%   Nummern ohne abschlie�enden Punkt
	12pt%                 Gr�ssere Schrift, besser lesbar am bildschrim
]{scrbook}

%
% Paket f�r �bersetzungen ins Deutsche
%
\usepackage[ngerman]{babel}

%
% Pakete um Latin1 Zeichnens�tze verwenden zu k�nnen und die dazu
% passenden Schriften.
%
\usepackage[latin1]{inputenc}
\usepackage[T1]{fontenc}

%
% Paket f�r Quotes
%
\usepackage[babel,german=swiss]{csquotes}

%
% Paket zum Erweitern der Tabelleneigenschaften
%
\usepackage{array}

%
% Paket f�r sch�nere Tabellen
%
\usepackage{booktabs}

%
% Paket um Grafiken einbetten zu k�nnen
%
\usepackage{graphicx}

%
% Spezielle Schrift im Koma-Script setzen.
%
\setkomafont{sectioning}{\normalfont\bfseries}
\setkomafont{captionlabel}{\normalfont\bfseries} 
\setkomafont{pagehead}{\normalfont\bfseries} % Kopfzeilenschrift
\setkomafont{descriptionlabel}{\normalfont\bfseries}

%
% Zeilenumbruch bei Bildbeschreibungen.
%
\setcapindent{1em}

%
% Kopf und Fu�zeilen
%
\usepackage{scrpage2}
\pagestyle{scrheadings}
% Inhalt bis Section rechts und Chapter links
\automark[section]{chapter}
% Mitte: leer
\chead{}

%
% mathematische symbole aus dem AMS Paket.
%
\usepackage{amsmath}
\usepackage{amssymb}

%
% Type 1 Fonts f�r bessere darstellung in PDF verwenden.
%
%\usepackage{mathptmx}           % Times + passende Mathefonts
%\usepackage[scaled=.92]{helvet} % skalierte Helvetica als \sfdefault
\usepackage{courier}            % Courier als \ttdefault

%
% Paket um Textteile drehen zu k�nnen
%
\usepackage{rotating}

%
% Paket f�r Farben im PDF
%
\usepackage{color}

%
% Paket f�r Links innerhalb des PDF Dokuments
%
\definecolor{LinkColor}{rgb}{0,0,0.5}
\usepackage[%
	pdftitle={Titel},% Titel der Diplomarbeit
	pdfauthor={Autor},% Autor(en)
	pdfcreator={LaTeX, LaTeX with hyperref and KOMA-Script},% Genutzte Programme
	pdfsubject={Betreff}, % Betreff
	pdfkeywords={Keywords}]{hyperref} % Keywords halt :-)
\hypersetup{colorlinks=true,% Definition der Links im PDF File
	linkcolor=LinkColor,%
	citecolor=LinkColor,%
	filecolor=LinkColor,%
	menucolor=LinkColor,%
	urlcolor=LinkColor}

%
% Paket um LIstings sauber zu formatieren.
%
\usepackage[savemem]{listings}
\lstloadlanguages{TeX}

%
% Listing Definationen f�r PHP Code
%
\definecolor{lbcolor}{rgb}{0.85,0.85,0.85}
\lstset{language=[LaTeX]TeX,
	numbers=left,
	stepnumber=1,
	numbersep=5pt,
	numberstyle=\tiny,
	breaklines=true,
	breakautoindent=true,
	postbreak=\space,
	tabsize=2,
	basicstyle=\ttfamily\footnotesize,
	showspaces=false,
	showstringspaces=false,
	extendedchars=true,
	backgroundcolor=\color{lbcolor}}
%
% ---------------------------------------------------------------------------
%

%
% Neue Umgebungen
%
\newenvironment{ListChanges}%
	{\begin{list}{$\diamondsuit$}{}}%
	{\end{list}}

%
% aller Bilder werden im Unterverzeichnis images gesucht:
%
\graphicspath{{images/}}

%
% Literaturverzeichnis-Stil
%
\bibliographystyle{plain}

%
% Anf�hrungsstriche mithilfe von \textss{-anzufuehrendes-}
%
\newcommand{\textss}[1]{"`#1"'}

%
% Strukturiertiefe bis subsubsection{} m�glich
%
\setcounter{secnumdepth}{3}

%
% Dargestellte Strukturiertiefe im Inhaltsverzeichnis
%
\setcounter{tocdepth}{3}

%
% Zeilenabstand wird um den Faktor 1.5 ver�ndert
%
%\renewcommand{\baselinestretch}{1.5}

%
% Neue Umgebungen
%
\newtheorem{theorem}{Satz}[chapter]
\newtheorem{definition}{Definition}[chapter]
\newtheorem{algorithm}{Verfahren}[chapter]

%
% Neue Mathe-Befehle
%
\newcommand{\R}{\mathbb{R}}
\newcommand{\F}{\mathcal{F}}
\newcommand{\nb}{\text{\em Nb. }}
\newcommand{\xopt}{x^*}
\newcommand{\A}{\mathcal{A}}
\newcommand{\I}{\mathcal{I}}

\usepackage{amsopn}
\DeclareMathOperator*{\kr}{Kern}
\DeclareMathOperator*{\im}{Im}
