\documentclass[%
	pdftex,%              PDFTex verwenden
	a4paper,%             A4 Papier
	oneside,%             Einseitig
	bibliography=totoc,%  Literaturverzeichnis einf�gen bibtotocnumbered: nummeriert
	listof=totoc,%		  Verzeichnisse einbinden in toc
	index=totoc,%         Index ins Verzeichnis einf�gen
	parskip=half,%        Europ�ischer Satz mit abstand zwischen Abs�tzen
	chapterprefix,%       Kapitel anschreiben als Kapitel
	headsepline,%         Linie nach Kopfzeile
	%footsepline,%        Linie vor Fusszeile
	%pointlessnumbers,%   Nummern ohne abschlie�enden Punkt
	12pt%                 Gr�ssere Schrift, besser lesbar am bildschrim
]{scrbook}

%
% Paket f�r �bersetzungen ins Deutsche
%
\usepackage[ngerman]{babel}

%
% Pakete um Latin1 Zeichens�tze verwenden zu k�nnen und die dazu
% passenden Schriften.
%
\usepackage[latin1]{inputenc}
\usepackage[T1]{fontenc}

%
% Paket um Latin Modern Font zu benutzen
%
\usepackage{lmodern}
\renewcommand*\familydefault{\sfdefault}% Verwende die Sans Version
\DeclareMathVersion{sansmath}% Mach die Mathe Font auch zu sans
\SetSymbolFont{letters}{sansmath}{OML}{cmbr}{m}{it}
\SetSymbolFont{operators}{sansmath}{OT1}{lmss}{m}{n}
\SetSymbolFont{symbols}{sansmath}{OMS}{cmbr}{m}{n}
\mathversion{sansmath}

%
% Paket f�r Quotes (Befehl: \enquote)
%
\usepackage[babel,german=quotes]{csquotes}

%
% Anf�hrungsstriche mithilfe von \textss{-anzufuehrendes-}
%
\newcommand{\textss}[1]{\enquote{#1}}% oder mit \glqq und \grqq

%
% Paket zum Erweitern der Tabelleneigenschaften
%
\usepackage{array}
\renewcommand{\arraystretch}{1.5}

%
% Paket f�r sch�nere Tabellen
%
\usepackage{booktabs}

%
% Paket f�r sch�nere Listen
%
\usepackage{paralist}
\setlength{\pltopsep}{1.5ex}% Abstand zwischen Text und Listenpunkt
\setlength{\plitemsep}{1.5ex}% Abstand zwischen Items

%
% Paket um Grafiken einzubetten
%
\usepackage{graphicx}

%
% Paket um Grafiken zu erstellen
%
\usepackage{tikz}

%
% Zeilenumbruch bei Bildbeschreibungen.
%
\setcapindent{1em}

%
% Kopf und Fu�zeilen
%
\usepackage{scrpage2}
\pagestyle{scrheadings}
% Inhalt bis Section rechts und Chapter links
\automark[section]{chapter}
% Mitte: leer
\chead{}

%
% mathematische symbole aus dem AMS Paket.
%
\usepackage{amsmath}
\usepackage{amssymb}

%
% Type 1 Fonts f�r bessere darstellung in PDF verwenden.
%
%\usepackage{mathptmx}           % Times + passende Mathefonts
%\usepackage[scaled=.92]{helvet} % skalierte Helvetica als \sfdefault
\usepackage{courier}            % Courier als \ttdefault

%
% Paket um Textteile drehen zu k�nnen
%
\usepackage{rotating}

%
% Paket f�r Farben im PDF
%
\usepackage{color}

%
% Paket f�r Links innerhalb des PDF Dokuments
%
%\definecolor{LinkColor}{rgb}{0,0,0.5}
\usepackage[%
	pdftitle={Vergleich zwischen halbglattem Newton-Verfahren und SQP-Verfahren},
	pdfauthor={Vicky Hartanto Tanzil},
	pdfcreator={LaTeX, LaTeX with hyperref and KOMA-Script},% Genutzte Programme
	pdfsubject={Bachelorarbeit}, % Betreff
	pdfkeywords={SQP, Semismooth-Newton, Optimization}]{hyperref}
%\hypersetup{colorlinks=true,% Definition der Links im PDF File
%	linkcolor=LinkColor,%
%	citecolor=LinkColor,%
%	filecolor=LinkColor,%
%	menucolor=LinkColor,%
%	urlcolor=LinkColor}

%
% Paket um LIstings sauber zu formatieren.
%
\usepackage[savemem]{listings}
\lstloadlanguages{TeX}

%
% Listing Definationen f�r PHP Code
%
\definecolor{lbcolor}{rgb}{0.85,0.85,0.85}
\lstset{language=[LaTeX]TeX,
	numbers=left,
	stepnumber=1,
	numbersep=5pt,
	numberstyle=\tiny,
	breaklines=true,
	breakautoindent=true,
	postbreak=\space,
	tabsize=2,
	basicstyle=\ttfamily\footnotesize,
	showspaces=false,
	showstringspaces=false,
	extendedchars=true,
	backgroundcolor=\color{lbcolor}}
%
% ---------------------------------------------------------------------------
%

%
% Neue Umgebungen
%
\newenvironment{ListChanges}%
	{\begin{list}{$\diamondsuit$}{}}%
	{\end{list}}

%
% aller Bilder werden im Unterverzeichnis images gesucht:
%
\graphicspath{{images/}}

%
% Literaturverzeichnis-Stil
%
\bibliographystyle{alpha}

%
% Strukturiertiefe bis subsubsection{} m�glich
%
\setcounter{secnumdepth}{3}

%
% Dargestellte Strukturiertiefe im Inhaltsverzeichnis
%
\setcounter{tocdepth}{1}

%
% Zeilenabstand wird um den Faktor 1.15 ver�ndert
%
%\renewcommand{\baselinestretch}{1.15}

%
% Neue Umgebungen
%
\newtheorem{theorem}{Satz}[chapter]
\newtheorem{definition}{Definition}[chapter]
\newtheorem{algorithm}{Verfahren}[chapter]
\newtheorem{testfunction}{Testfunktion}%[chapter]
\newtheorem{testproblem}{Testproblem}%[chapter]
\newtheorem{example}{Beispiel}[chapter]

%
% Neue Mathe-Befehle
%
\newcommand{\R}{\mathbb{R}}
\newcommand{\F}{\mathcal{F}}
\newcommand{\xopt}{x^*}
\newcommand{\Lagrange}{\mathcal{L}}
\newcommand{\A}{\mathcal{A}}
\newcommand{\I}{\mathcal{I}}

%
% Neue Mathe-Operatoren
%
\usepackage{amsopn}
\DeclareMathOperator{\nb}{Nb.\ }
\DeclareMathOperator{\kr}{Kern}
\DeclareMathOperator{\im}{Im}
\DeclareMathOperator{\co}{co}
